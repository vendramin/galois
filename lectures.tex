\RequirePackage{amsmath} 
\documentclass[graybox]{svmult}

\usepackage{type1cm}        % activate if the above 3 fonts are
                            % not available on your system
                            
\usepackage{makeidx}         % allows index generation
\usepackage{graphicx}        % standard LaTeX graphics tool
                             % when including figure files
\usepackage{multicol}        % used for the two-column index
\usepackage[bottom]{footmisc}% places footnotes at page bottom

\usepackage[T1]{fontenc}
\usepackage[utf8]{inputenc}
\usepackage{amsmath}
\usepackage{anyfontsize}
\usepackage{float}
\usepackage{wrapfig}
\usepackage{amssymb}
\usepackage{amstext}
\usepackage{mathtools}
\usepackage{stmaryrd}
\usepackage{xcolor} 
\usepackage{centernot}
\usepackage{listings}
\usepackage{multicol}
\usepackage{mathptmx}
%\usepackage{newtxtext,newtxmath}
\usepackage{datetime}
\usepackage{stmaryrd}
\usepackage{tikz-cd}
\usepackage{quiver}
\usepackage{listings}
\usepackage{helvet}
\usepackage{courier}
\usepackage{type1cm}         
\usepackage{makeidx}        
\usepackage{graphicx}        
\usepackage{multicol}        
\usepackage{hyperref} 
\usepackage{colortbl}
\usepackage{chngcntr}

\lstdefinelanguage{Julia}%
  {morekeywords={abstract,break,case,catch,const,continue,do,else,elseif,%
      end,export,false,for,function,immutable,import,importall,if,in,%
      macro,module,otherwise,quote,return,switch,true,try,type,typealias,%
      using,while},%
   sensitive=true,%
   alsoother={$},%
   morecomment=[l]\#,%
   morecomment=[n]{\#=}{=\#},%
   morestring=[s]{"}{"},%
   morestring=[m]{'}{'},%
}[keywords,comments,strings]%

\definecolor{background}{HTML}{F5F5F5}
\definecolor{jlstring}{HTML}{880000}%          % julia's strings
\definecolor{jlbase}{HTML}{444444}%            % julia's base color
\definecolor{jlkeyword}{HTML}{444444}%         % julia's keywords
\definecolor{jlliteral}{HTML}{78A960}%         % julia's literals
\definecolor{jlbuiltin}{HTML}{397300}%         % julia's built-ins
\definecolor{jlmacros}{HTML}{1F7199}%          % julia's macros
\definecolor{jlfunctions}{HTML}{444444}%       % julia's functions
\definecolor{jlcomment}{HTML}{888888}%         % julia's comments
\definecolor{jlstring}{HTML}{880000}%          % julia's strings


\lstset{%
    language         = Julia,
    basicstyle       = \color{jlstring}\ttfamily\scriptsize,
    backgroundcolor  = \color{background},
    keywordstyle     = \color{jlkeyword},
    stringstyle      = \color{jlstring},
    commentstyle     = \color{jlcomment},
    showstringspaces = false,
    columns=fixed,
}

% change numbers 
\let\remark\relax
\let\theorem\relax
\let\lemma\relax
\let\definition\relax
\let\proposition\relax
\let\corollary\relax
\let\exercise\relax
\let\example\relax
\let\conjecture\relax

% % Numerar con sección y no resetear al cambiar de capítulo
% \counterwithout{subsection}{section}
\counterwithout{theorem}{section}
\spnewtheorem{theorem}{\theoremname}[section]{\bfseries}{\itshape}

\renewcommand\thetheorem{\thesection.\arabic{theorem}}
\spnewtheorem{lemma}[theorem]{\lemmaname}{\bfseries}{\itshape}
\spnewtheorem{definition}[theorem]{\definitionname}{\bfseries}{\upshape}
\spnewtheorem{proposition}[theorem]{\propositionname}{\bfseries}{\itshape}
\spnewtheorem{corollary}[theorem]{\corollaryname}{\bfseries}{\itshape}
\spnewtheorem{exercise}[theorem]{\exercisename}{\bfseries}{\upshape}
\spnewtheorem{example}[theorem]{\examplename}{\bfseries}{\upshape}
\spnewtheorem{examples}[theorem]{\examplesname}{\bfseries}{\upshape}
\spnewtheorem{remark}[theorem]{\remarkname}{}{\upshape}
\spnewtheorem{conjecture}[theorem]{\conjecturename}{\bfseries}{\upshape}
\spnewtheorem{notation}[theorem]{\notationname}{\bfseries}{\upshape}
\spnewtheorem{steps}[theorem]{\stepsname}{\bfseries}{\upshape}
\spnewtheorem{convention}[theorem]{\conventionname}{\bfseries}{\upshape}

% % Numerar con sección y no resetear al cambiar de capítulo
%\counterwithout{section}{chapter}

% % No sections in TOC
% \setcounter{secnumdepth}{1}
% \setcounter{tocdepth}{0}

  \usepackage{titlesec}
  \titleformat{\section}
    {\secsize\secstyle}{Lecture \thesection.}{1em}{}

  \titleformat{\subsection}
    {\secsize\secstyle}{\S\thesubsection.}{1em}{}


% para enumerar
\renewcommand{\labelenumi}{\textbf{\arabic{enumi})}}

\makeindex             

\newcommand{\Gal}{\operatorname{Gal}}
\renewcommand{\I}{\operatorname{I}}
\newcommand{\II}{\operatorname{II}}

\newcommand{\GAP}{\textsf{GAP}}
\newcommand{\FK}{\mathcal{E}}
\newcommand{\ad}[1]{\operatorname{ad}\,#1}

%\newcommand{\N}{\mathbb{N}}
\newcommand{\Q}{\mathbb{Q}}
\newcommand{\Z}{\mathbb{Z}}
\newcommand{\F}{\mathbb{F}}
\newcommand{\R}{\mathbb{R}}
\newcommand{\C}{\mathbb{C}}
\renewcommand{\H}{\mathbb{H}}
\newcommand{\A}{\mathbb{A}}
\newcommand{\K}{\mathbb{K}}
\newcommand{\T}{\mathbb{T}}
\renewcommand{\D}{\mathbb{D}}
\newcommand{\B}{\mathbb{B}}
\newcommand{\Fun}{\operatorname{Fun}}
\newcommand{\mpl}{\operatorname{mpl}}
\newcommand{\cL}{\mathcal{L}}
\newcommand{\cE}{\mathcal{E}}
\newcommand{\cH}{\mathcal{H}}

\newcommand{\GF}{\mathsf{GF}}
\newcommand{\MAX}{\operatorname{MAX}}
\newcommand{\MIN}{\operatorname{MIN}}
\newcommand{\cf}{\operatorname{cf}}
\newcommand{\cl}{\operatorname{cl}}
\newcommand{\cd}{\operatorname{cd}}
\newcommand{\bL}{\mathbf{L}}
\newcommand{\bP}{\mathbf{P}}

\newcommand{\Nil}{\operatorname{Nil}}
\newcommand{\rad}{\operatorname{rad}}
\newcommand{\rank}{\operatorname{rank}}
\newcommand{\norm}{\operatorname{norm}}

\newcommand{\Aff}{\mathrm{Aff}}
\newcommand{\Ann}{\operatorname{Ann}}
\newcommand{\Der}{\operatorname{Der}}
\newcommand{\Core}{\operatorname{Core}}
\newcommand{\Soc}{\operatorname{Soc}}
\newcommand{\Fix}{\operatorname{Fix}}
\newcommand{\Rad}{\mathrm{rad}}
\newcommand{\Inn}{\mathrm{Inn}}
\newcommand{\dist}{\mathrm{dist}}
\newcommand{\Out}{\mathrm{Out}}
\newcommand{\Ext}{\mathrm{Ext}}
\newcommand{\Img}{\mathrm{im}}
\newcommand{\Hol}{\operatorname{Hol}}
\newcommand{\Hom}{\operatorname{Hom}}
\newcommand{\Alg}{\operatorname{Alg}}
\newcommand{\Bil}{\operatorname{Bil}}
\newcommand{\op}{\operatorname{op}}
\newcommand{\gr}{\operatorname{gr}}
\newcommand{\Syl}{\mathrm{Syl}}
\newcommand{\id}{\operatorname{id}}
\newcommand{\Aut}{\operatorname{Aut}}
\newcommand{\End}{\operatorname{End}}
\newcommand{\Irr}{\operatorname{Irr}}
\newcommand{\Alt}{\mathbb{A}}
\newcommand{\Sym}{\mathbb{S}}
\newcommand{\lcm}{\mathrm{mcm}}
\newcommand{\diag}{\operatorname{diag}}
\newcommand{\spec}{\operatorname{Spec}}
\newcommand{\supp}{\operatorname{supp}}
\newcommand{\trace}{\operatorname{trace}}
\newcommand{\sgn}{\operatorname{sign}}
\newcommand{\ch}{\operatorname{char}}

\newcommand{\inner}{\operatorname{inn}}
\newcommand{\ext}{\operatorname{ext}}
\newcommand{\im}{\operatorname{im}}
\newcommand{\Ret}{\operatorname{Ret}}

\newcommand{\GL}{\mathbf{GL}}
\newcommand{\SL}{\mathbf{SL}}
\newcommand{\PSL}{\mathbf{PSL}}
\newcommand{\PGL}{\mathbf{PGL}}

% Table of contents for lectures and topics
\makeatletter
\newcommand\listtopicsname{Table of contents}
\newcommand\listoftopics{
    \chapter*{\listtopicsname}\@starttoc{top}}
\makeatother

\usepackage{newtxtext}       % 
\usepackage[varvw]{newtxmath}       % selects Times Roman as basic font

% see the list of further useful packages
% in the Reference Guide

\makeindex             % used for the subject index
                       % please use the style svind.ist with
                       % your makeindex program

%%%%%%%%%%%%%%%%%%%%%%%%%%%%%%%%%%%%%%%%%%%%%%%%%%%%%%%%%%%%%%%%%%%%%%%%%%%%%%%%%%%%%%%%%

\begin{document}

\title*{Galois theory}
% Use \titlerunning{Short Title} for an abbreviated version of
% your contribution title if the original one is too long
\author{Leandro Vendramin\orcidID{0000-0003-0954-7785}}
% Use \authorrunning{Short Title} for an abbreviated version of
% your contribution title if the original one is too long
\institute{Leandro Vendramin \at Vrije Universiteit Brussel, \email{Leandro.Vendramin@vub.be}}
%
% Use the package "url.sty" to avoid
% problems with special characters
% used in your e-mail or web address
%
\maketitle

\abstract{The notes correspond to the bachelor 
course \emph{Galois theory} of the 
Vrije Universiteit Brussel, 
Faculty of Sciences, 
Department of Mathematics and Data Sciences. The course
is divided into thirteen two-hour lectures. 
\newline\indent 
The material is somewhat standard. Basic texts on fields and Galois theory 
are for example \cite{MR1645586} and 
\cite{MR3379917}. 
As usual, we also mention a set of 
\href{https://kconrad.math.uconn.edu/blurbs/}{great expository papers} by 
Keith Conrad, the notes are extremely well-written and useful  
at every stage of a mathematical career. 
Several chapters contain optional paragraphs that give examples of 
how to apply \href{https://oscar.computeralgebra.de/}{OSCAR Computer Algebra System}
to concrete problems in Galois theory. 
\newline\indent 
Thanks go to Wouter Appelmans, Luca Descheemaeker, 
Alejandro de la Cueva Merino, 
Wannes Malfait,
Lukas Simons. 
\newline\indent  
This version 
was compiled on \today~at~\currenttime.}

\tableofcontents 

\chapter{}

\topic{Fields}

Recall that a \textbf{field} is a commutative
ring such that $1\ne 0$ and 
that every non-zero element is invertible. Examples
of (infinite) fields are $\Q$, $\R$ and $\C$. If $p$
is a prime number, then $\Z/p$ is a field. 

\begin{example}
	The abelian group $\Z/2\times\Z/2$ is a field
	with multiplication
	\[
		(a,b)(c,d)=(ac+bd,ad+bc+bd).
	\]
\end{example}

\begin{example}
	$\Q(i)=\{a+bi:a,b\in\Q\}$ and 
	$\Q(\sqrt{2})$ are fields.
\end{example}

\[
\begin{tikzcd}
	{\mathbb{C}} \\
	&& {\mathbb{R}} \\
	{\mathbb{Q}(i)} && {\mathbb{Q}(\sqrt{2})} \\
	& {\mathbb{Q}}
	\arrow[no head, from=3-3, to=4-2]
	\arrow[no head, from=2-3, to=3-3]
	\arrow[no head, from=1-1, to=3-1]
	\arrow[no head, from=3-1, to=4-2]
	\arrow[no head, from=1-1, to=2-3]
\end{tikzcd}
\]


\begin{exercise}
	\label{xca:Q(i)}
	Prove that $\Q(i)$ and $\Q(\sqrt{2})$ are not isomorphic as fields.
\end{exercise}

If $R$ is a ring, there exists a unique ring homomorphism
$\Z\to R$, $m\mapsto m1$. The image $\{m1:m\in\Z\}$ 
of this homomorphism is a subring 
of $R$ and it is known as the \textbf{ring of integers} of $R$. The
kernel is a subgroup of $\Z$ and is generated by
some $t\geq0$. The integer $t$ is 
the \textbf{characteristic} of the ring $R$. 

\begin{exercise}
	The characteristic of a field is either zero or
	a prime number. 
\end{exercise}

Recall that a commutative ring $R$ is an \textbf{integral 
domain} if $xy=0\implies x=0$ or $y=0$. Fields
are integral domains. 

\begin{exercise}
	Let $K$ be a field. Prove that
	the following statements are equivalent:
	\begin{enumerate}
		\item $K$ is of characteristic zero.
		\item The additive order of $1$ is infinite. 
		\item The additive order of each $x\ne0$ is infinite.
		\item The ring of integers of $K$ is isomorphic to $\Z$.
	\end{enumerate}
\end{exercise}

\begin{exercise}
	Let $K$ be a field. Prove that
	the following statements are equivalent:
	\begin{enumerate}
		\item $K$ is of characteristic $p$.
		\item The additive order of $1$ is $p$. 
		\item The additive order of each $x\ne0$ is $p$.
		\item The ring of integers of $K$ is isomorphic to $\Z/p$.
	\end{enumerate}
\end{exercise}

% The following exercise is important. 

% \begin{exercise}
% 	Prove that if $K$ is a finite field, then
% 	$|K|=p^m$ for some prime number $p$ and some $m\geq1$. 
% \end{exercise}

\begin{definition}
	\index{Subfield}
	A \textbf{subfield} of a ring $R$ is a subring of $R$ 
	that is also a field.
\end{definition}

Note that if $K$ is a subfield of $E$, then
the characteristic of $K$ coincides
with the characteristic 
of $E$. Moreover, if $K\to L$ is a field homomorphism, then
$K$ and $L$ have the same characteristic. 

\begin{exercise}
	Let $K$ be a field of characteristic $p$. Prove
	that $K\to K$, $x\mapsto x^{p^n}$, is a field homomorphism
	for all $n\in\Z_{\geq 0}$. 
\end{exercise}

Note that finite fields are of characteristic $p$. 

Let $K$ be a subfield of a field $E$. Then $E$ 
is a $K$-vector space with the usual scalar multiplication
$K\times E\to E$, 
$(\lambda, x)\mapsto \lambda x$.

\begin{definition}
	A field $K$ is \textbf{prime} if there are no
	proper subfields of $K$. 
\end{definition}

Examples of prime fields are $\Q$ and $\Z/p$ for a prime number $p$.

\begin{proposition}
	Let $K$ be a field. The following statements hold:
	\begin{enumerate}
		\item $K$ contains a unique prime field, it is known as the 
			\textbf{prime subfield} of $K$.
		\item The prime subfield of $K$ is either isomorphic to $\Q$ if 
			the characteristic of $K$ is zero, or it is isomorphic to $\Z/p$ for
			some prime number $p$ if the characteristic of $K$ is $p$. 
	\end{enumerate}
\end{proposition}

\begin{proof}
	To prove the first claim let $L$ be the intersection
	of all the subfields of $K$. Then $L$ is a subfield of $K$. 
	If $F$ is a subfield of $L$, then $F$ is a subfield
	of $K$. Thus $L\subseteq F$ and hence $F=L$, which proves
	that $L$ is prime. If $L_1$ is a subfield of $K$
	and $L_1$ is prime, then $L\subseteq L_1$ and 
	hence $L=L_1$. 

	Let $K_0$ be the prime field of $K$. Suppose that $K$ is of characteristic
	$p>0$. Then the ring $K_\Z$ of integers of $K$ 
	is a field isomorphic to $\Z/p$ and hence $K_0\simeq
	K_\Z$. Suppose now that the characteristic of $K$ is zero. Let
	$E=\{m1/n1:m,n\in\Z,n\ne 0\}$. We claim that $K_0=E$. Since $K_\Z\subseteq
	K_0$, it follows that $E\subseteq K_0$. Hence $E=K_0$, as $E$ is a subfield
	of $K$.  
\end{proof}

\begin{definition}
	Let $E$ be a field and $K$ be a subfield of $E$. Then 
	$E$ is a \textbf{field extension} of $K$. We will use
	the notation $E/K$. 
\end{definition}

If $E$ is an extension of $K$, then $E$ is a
$K$-vector space. 

\begin{definition}
	The degree of an extension $E$ of $K$ 
	is the integer $\dim_KE$. It will be denoted by $[E:K]$. 
\end{definition}

We say that $E$ is a finite extension of $K$ 
if $[E:K]$ is finite. 

\begin{example}
	Let $K$ be a field. Then $[K:K]=1$. Conversely, 
	if $E$ is an extension of $K$ and $[E:K]=1$, then $K=E$. 
	If not, let $x\in E\setminus K$. We claim that
	$\{1,x\}$ is linearly independent over $K$. Indeed, 
	if $a1+bx=0$ for some $a,b\in K$, then $bx=-a$. If 
	$b\ne 0$, then $x=-a/b\in K$, a contradiction. If $b=0$, then 
	$a=0$. 
\end{example}

We know that $[\C:\R]=2$. 

\begin{example}
	A basis of $\Q(\sqrt{2})$ over $\Q$ 
	is given by $\{1,\sqrt{2}\}$. Then 
	$[\Q(\sqrt{2}):\Q]=2$. The calculations 
	can be easily done by computer: 
\begin{lstlisting}
julia> E, a = quadratic_field(2)
(Real quadratic field defined by x^2 - 2, sqrt(2))
julia> characteristic(E)
0
julia> K = prime_field(E)
Rational Field
julia> degree(E)
2
julia> basis(E)
2-element Vector{nf_elem}:
 1
 sqrt(2)
julia> one(K)==one(E)
true
julia> zero(K)==zero(E)
true
\end{lstlisting}
\end{example}

\begin{example}
	Since $\Q$ is numerable and 
	$\R$ is not, $[\R:\Q]>\aleph_0$. If $\{x_i:i\in\Z_{>0}\}$ 
	is a numerable basis of $\R$ over $\Q$, for each
	$n$ consider the $\Q$-vector space
	$V_n$ generated by $\{x_1,\dots,x_n\}$. Then 
	\[
		\R=\bigcup_{n\geq1}V_n,
	\]
	is numerable, as each $V_n$ is numerable, a contradiction.
\end{example}

If $E$ is an extension of $K$ and $E$ is finite,
then $[E:K]$ is finite. 

\begin{proposition}
	Let $K$ be a finite field. Then $|K|=p^m$ 
	for some prime number $p$ and some $m\geq1$. 
\end{proposition}

\begin{proof}
	We know the prime subfield $K_0$ of $K$ is isomorphic to $\Z/p$. 
	In particular, $|K_0|=p$. Since $K$ is finite, 
	$[K:K_0]=m$ for some $m$. If $\{x_1,\dots,x_m\}$ is a basis
	of $K$ over $K_0$, then each element
	of $K$ can be written uniquely as
	$\sum_{i=1}^ma_ix_i$ for some $a_1,\dots,a_m\in K_0$. Then
	$K\simeq K_0^m$ and hence $|K|=|K_0^m|=p^m$. 
\end{proof}

We now perform some basic calculations 
with a finite field of eight elements: 
\begin{lstlisting}
julia> E, x = FiniteField(2, 3, "x")
(Finite field of degree 3 over F_2, x)
julia> characteristic(E)
2
julia> prime_field(E)
Galois field with characteristic 2
julia> degree(E)
3
julia> size(E)
8
julia> [z for z in E]
8-element Vector{fq_nmod}:
 0
 1
 x
 x + 1
 x^2
 x^2 + 1
 x^2 + x
 x^2 + x + 1
\end{lstlisting}

%julia> F, y = FiniteField(5, 2, "y")
%(Finite field of degree 2 over F_5, y)
%julia> degree(F)
%2
%julia> prime_field(F)
%Galois field with characteristic 5
%julia> size(F)
%25


\begin{definition}
	Let $E$ be an extension of $K$. A \textbf{subextension} $F/K$ 
	of $E/K$ is a subfield $F$ of $E$ that contains $K$, that is
	$K\subseteq F\subseteq E$. 
\end{definition}

\begin{definition}
	Let $E$ and $E_1$ be extensions over $K$. An extension
	\textbf{homomorphism} $E\to E_1$ is a 
	field homomorphism $\sigma\colon E\to E_1$ such that 
	$\sigma(x)=x$ for all $x\in K$. 
\end{definition}

To describe the homomorphism $\sigma\colon E\to E_1$ of the extensions over $K$
one typically writes the commutative diagram 
\[
	\begin{tikzcd}
	K & K \\
	E & {E_1} 
	\arrow["\sigma", from=2-1, to=2-2]
	\arrow[equal, no head, from=1-1, to=1-2]
	\arrow[hook, from=1-1, to=2-1]
	\arrow[hook, from=1-2, to=2-2]
\end{tikzcd}
\]
We write $\Hom(E/K,E_1/K)$ to denote
the set of homomorphism $E\to E_1$ of extensions of $K$. Note
that if $\sigma\in\Hom(E/K,E_1/K)$, then
$\sigma$ is a $K$-linear map, as
\[
	\sigma(\lambda x)=\sigma(\lambda)\sigma(x)=\lambda\sigma(x)
\]
for all $\lambda\in K$ and $x\in E$. 

\begin{example}
	The conjugation map $\C\to\C$, $z\mapsto\overline{z}$, 
	is an endomorphism of $\C$ as an extension over $\R$. Let 
	$\varphi\in\Hom(\C/\R,\C/\R)$. Then 
	\[
	\varphi(x+iy)=\varphi(x)+\varphi(i)\varphi(y)=x+\varphi(i)y
	\]
	for all $x,y\in\R$. Since $\varphi(i)^2=\varphi(i^2)=\varphi(-1)=-1$, 
	it follows that $\varphi(i)\in\{-i,i\}$. Thus either 
	$\varphi(x+iy)=x+iy$ or $\varphi(x+iy)=x-iy$. 
\end{example}

\begin{exercise}
	Prove that if $K$ is a field and $\sigma\colon K\to K$ is a field homomorphism, 
then $\sigma\in\Hom(K/K_0,K/K_0)$. 
\end{exercise}

If $E/K$ is an extension, then
\[
	\Aut(E/K)=\{\sigma:\sigma\colon E\to E\text{ is a bijective extension homomorphism}\}
\]
is a group with composition.

\begin{definition}
	Let $E/K$ be an extension. The \textbf{Galois group}
	of $E/K$ is the group
	$\Aut(E/K)$ and it will be denoted by $\Gal(E/K)$. 
\end{definition}

A typical example: $\Gal(\C/\R)\simeq\Z/2$. 

As an example, we show with the computer that $\Gal(\Q(\sqrt{2})/\Q)\simeq\Z/2$:
\begin{lstlisting}
julia> E, x = quadratic_field(2)
(Real quadratic field defined by x^2 - 2, sqrt(2))
julia> characteristic(E)
0
julia> G, C = galois_group(E);
julia> describe(G)
"C2"
julia> order(G)
2
\end{lstlisting}

\begin{example}
	Let $\theta=\sqrt[3]{2}$ and let $E=\{a+b\theta+c\theta^2:a,b,c\in\Q\}$. Note that 
\[
	a+b\theta+c\theta^2=0 \Longleftrightarrow a=b=c=0. 
\]
% In fact, if $abc\ne 0$, then $aX^2+bX+c\ne 0$ and 
% thus $X^3-2=q(X)(aX^2+bX+c)+r(X)$ for some polynomials
% $q(X)\in\Q[X]$ and $r(X)=eX+f\in\Q[X]$. Evaluate in $\theta$ 
% to obtain that $r(\theta)=0$ and hence $r(X)=0$ in $\Q[X]$. This implies  
% that $aX^2+bX+c$ divides $X^3-2$, a contradiction since
% $X^3-2$ is irreducible in $\Q[X]$. 
Then $E$ is an extension of $\Q$ such that $[E:\Q]=3$. We claim
that $\Gal(E/\Q)$ is trivial. If 
$\sigma\in\Gal(E/\Q)$ and $z=a+b\theta+c\theta^2$, then
$\sigma(z)=a+b\sigma(\theta)+c\sigma^2(\theta)$. Since
$\sigma(\theta)^3=\sigma(\theta^3)=\sigma(2)=2$, it follows
that $\sigma(\theta)=\theta$ and therefore
$\sigma=\id$. 
\end{example}

\begin{exercise}
    Prove that the polynomial $X^3-2$ is irreducible in $\Q[X]$.  
\end{exercise}

The previous exercise can easily be solved using
computers: 
\begin{lstlisting}
julia> R, x = PolynomialRing(QQ, "x");
julia> is_irreducible(x^3-2)
true
\end{lstlisting}

The following exercise is known as the 
\emph{Eisenstein's irreducibility criterion}:

\begin{exercise}
    \index{Eisenstein's criterion}    
    Let $A$ be a unique factorization domain and $K$ be its fraction field. 
    Let $f=\sum_{i=0}^n a_iX^i\in K[X]$ be a polynomial of degree $n>0$. 
    Assume that there exists a prime element $p\in A$ such that
    $p\mid a_i$ for all $i\in\{0,1,\dots,n-1\}$, $p\nmid a_n$ and
    $p^2\nmid a_0$. Then $f$ is irreducible in $K[X]$. 
\end{exercise}

\begin{exercise}
    Prove that
    the polynomials 
    $f=X^{10}+60X^7+82X^6-36X^3+2$ and 
    $g=3X^{10}+15X^2-45$ are irreducible in $\Z[X]$. 
\end{exercise}

\begin{exercise}
    Is the polynomial $f=3(X^{10}+5X^2-15)$ irreducible in $\Z[X]$? 
\end{exercise}

If $E/K$ is an extension and $S$ is a subset of $E$, then
there exists a unique smallest 
subextension $F/K$ of $E/K$ such that
$S\subseteq F$. In fact, 
\[
	F=\bigcap\{T:\text{$T$ is a subfield of $E$ that contains $K\cup S$}\} 
\]
If $L/K$ is a subextension of $E/K$ such that 
$S\subseteq L$, then $F\subseteq L$ by definition. The 
extension $F$ is known as the \textbf{subextension generated by} 
$S$ and
it will be denoted by $K(S)$. 
If $S=\{x_1,\dots,x_n\}$ is finite,
then $K(S)=K(x_1,\dots,x_n)$ is said to be of \textbf{finite type}. 

\begin{example}
	If $\{e_1,\dots,e_n\}$ is a basis of $E$ over $K$, 
	then $E=K(e_1,\dots,e_n)$. 
\end{example}

\begin{example}
	The field $\Q(\sqrt{2})$ is precisely the extension 
	of $\R/\Q$ generated by $\sqrt{2}$. 
\end{example}

Let $E/K$ be an extension and $S$ and $T$ be subsets of $E$.
Then 
\[
	K(S\cup T)=K(S)(T)=K(T)(S).
\]
If, moreover, 
$S\subseteq T$, then $K(S)\subseteq K(T)$. 

\topic{Algebraic extensions}

\begin{definition}
\index{Algebraic!element}
\index{Trascendental!element}
	Let $E/K$ be an extension. An element $x\in E$
	is \textbf{algebraic} over $K$ if there
	exists a non-zero polynomial 
	$f(X)\in K[X]$ such that $f(x)=0$. If $x$ is
	not algebraic over $K$, 
	then it is called \textbf{trascendental} over $K$.
\end{definition}

If $E/K$ is an extension, let 
\[
	\overline{K}_E=\{x\in E:x\text{ is algebraic over }K\}. 
\]
%is the \textbf{algebraic closure} of $K$ in $E$. 

\begin{definition}	
\index{Algebraic!extension}
	An extension $E/K$ is \textbf{algebraic} if 
	every $x\in E$ is algebraic over $K$. 
\end{definition}

If $K$ is a field, every $x\in K$ is algebraic over $K$,
as $x$ is a root of $X-x\in K[X]$. In particular, $K/K$ is
an algebraic extension. 

\begin{example}
	$\C/\R$ is an algebraic extension. If $z\in\C\setminus\R$, then
	$z$ is a root of the polynomial 
	$X^2+(z+\overline{z})X+|z|^2\in\R[X]$. 
\end{example}

If $F/K$ is an algebraic extension $x\in E$ is algebraic
over $K$ for some field $E\supseteq F$, 
then $x$ is algebraic over $F$. 

\begin{example}
	$\Q(\sqrt{2})/\Q$ is algebraic, as the number
	$a+b\sqrt{2}$ is a root of the polynomial
	$X^2-2aX+(a^2-2b^2)\in\Q[X]$. 
\end{example}

\index{Lindemann's theorem}
\index{Hermite's theorem}
The extension $\C/\Q$ is not algebraic. For example, Hermite proved 
that $e$ is transcendental 
over $\Q$; see \cite[Therem 24.4]{MR3379917}. Lindemann's theorem 
states that $\pi$ is 
not algebraic $\Q$; see \cite[Theorem 24.5]{MR3379917}. 

\chapter{}

If $E/K$ is an extension and $S$ is a subset of $E$, then
there exists a unique smallest 
subextension $F/K$ of $E/K$ such that
$S\subseteq F$. In fact, 
\[
	F=\bigcap\{T:\text{$T$ is a subfield of $E$ that contains $K\cup S$}\} 
\]
If $L/K$ is a subextension of $E/K$ such that 
$S\subseteq L$, then $F\subseteq L$ by definition. The 
extension $F$ is known as the \textbf{subextension generated by} 
$S$ and
it will be denoted by $K(S)$. 
If $S=\{x_1,\dots,x_n\}$ is finite,
then $K(S)=K(x_1,\dots,x_n)$ is said to be of \textbf{finite type}. 

\begin{example}
	If $\{e_1,\dots,e_n\}$ is a basis of $E$ over $K$, 
	then $E=K(e_1,\dots,e_n)$. 
\end{example}

\begin{example}
	The field $\Q(\sqrt{2})$ is precisely the extension 
	of $\R/\Q$ generated by $\sqrt{2}$. 
\end{example}

Let $E/K$ be an extension and $S$ and $T$ be subsets of $E$.
Then 
\[
	K(S\cup T)=K(S)(T)=K(T)(S).
\]
If, moreover, 
$S\subseteq T$, then $K(S)\subseteq K(T)$. 

\topic{Algebraic extensions}

\begin{definition}
	Let $E/K$ be an extension. An element $x\in E$
	is \textbf{algebraic} over $K$ if there
	exists a non-zero polynomial 
	$f(X)\in K[X]$ such that $f(x)=0$. If $x$ is
	not algebraic over $K$, 
	then it is called \textbf{trascendent} over $K$.
\end{definition}

If $E/K$ is an extension, then 
\[
	\overline{K}_E=\{x\in E:x\text{ is algebraic over }K\}
\]
is the \textbf{algebraic closure} of $K$ in $E$. 

\begin{definition}	
	An extension $E/K$ is \textbf{algebraic} if 
	every $x\in E$ is algebraic over $K$. 
\end{definition}

If $K$ is a field, every $x\in K$ is algebraic over $K$,
as $x$ is a root of $X-x\in K[X]$. In particular, $K/K$ is
an algebraic extension. 

\begin{example}
	$\C/\R$ is an algebraic extension. If $z\in\C\setminus\R$, then
	$z$ is a root of the polynomial 
	$X^2+(z+\overline{z})X+|z|^2\in\R[X]$. 
\end{example}

If $F/K$ is an algebraic extension and $x\in E$ is algebraic
over $K$, then $x$ is algebraic over $E$. 

\begin{example}
	$\Q(\sqrt{2})/\Q$ is algebraic, as the number
	$a+b\sqrt{2}$ is a root of the polynomial
	$X^2-2aX+(a^2-2b^2)\in\Q[X]$. 
\end{example}

The extension $\C/\Q$ is not algebraic. 

If $E/K$ is an extension and $x\in E$ is algebraic
over $K$, then the evaluation homomorphism 
$K[X]\to E$, $f\mapsto f(x)$, is not injective. In particular,
its kernel is a non-zero ideal and hence 
it is generated by a monic polynomial $f$. 

\begin{definition}
	Let $E/K$ be an extension and $x\in E$ be an algebraic element.  The monic
	polynomial that generates the kernel of $K[X]\to E$, $f\mapsto f(x)$, is
	known as the \textbf{minimal polynomial} of $x$ over $K$ and it will be
	denoted by $f(x,K)$. The \textbf{degree} of $x$ over $K$ is then $\deg
	f(x,K)$. 
\end{definition}

Some basic properties of the minimal polynomial of an algebraic element:

\begin{proposition}
	Let $E/K$ be an extension and $x\in E$. 
	\begin{enumerate}
		\item If $g\in K[X]$ is such that $g(x)=0$, then $f(x,K)$ divides $g$. 
		\item If $g(x)=0$ and $g\ne 0$, then $\deg g\geq\gr f(x,K)$.
		\item $f(x,K)$ is irreducible in $K[X]$.
		\item If $g(x)=0$ and $g(X)$ is monic and irreducible, then
			$g=f(x,K)$. 
		\item If $F/K$ is a subextension of $E/K$, then $f(x,F)$ divides
			$f(x,K)$. 
	\end{enumerate}
\end{proposition}

\begin{proof}
	Write $f=f(x,K)$ to denote the minimal polynomial of $x$. 
	To prove 1) note that $g(x)=0$ implies that	$g$ belongs to the kernel of
	the evaluation map, so $g$ is a multiple of $f$. Now 2) follows from
	1). To prove 3) note that if $f=gh$ for some $g,h\in K[X]$ such that
	$0<\deg g,\deg h<\deg f$, then $f(x)=0$ implies that 
	either $g(x)=0$ or $h(x)=0$, a
	contradiction. 4) is trivial. Finally we prove 5). Since $f\in K[X]\subseteq F[X]$ 
	and $f(x)=0$, it follows from 3) that $f(x,F)$ divides $f$. 
\end{proof}

Some easy examples: $f(i,\R)=X^2+1$ and 
$f(\sqrt[3]{2},\Q)=X^3-2$. 

\begin{example}
	Let us compute 
	$f(\sqrt{2}+\sqrt{3},\Q)$. Let $\alpha=\sqrt{2}+\sqrt{3}$. 
	Then 
	\begin{align*}
		\alpha-\sqrt{2}=\sqrt{3} & \implies 
		(\alpha-\sqrt{2})^2=3 \implies \alpha^2-2\sqrt{2}\alpha+2=3\\
		&\implies \alpha^2-1=2\sqrt{2}\alpha \implies
		(\alpha^2-1)^2=8\alpha^2\implies
		\alpha^4-10\alpha^2+1=0.
	\end{align*}
	Thus $\alpha$ is a root of $g=X^4-10X^2+1$. To prove that $g=f(\alpha,\Q)$ 
	it is enough to prove that 
	$g$ is irreducible in $\Q[X]$. First note that 
	the roots
	of $g$ are $\sqrt{2}+\sqrt{3}$, $\sqrt{2}-\sqrt{3}$, 
	$-\sqrt{2}+\sqrt{3}$ and $-\sqrt{2}-\sqrt{3}$. This means that
	if $g$ is not irreducible, 
	then $g=hh_1$ for some polynomials $h,h_1\in\Q[X]$ such that
	$\deg h=\deg h_1=2$. This is not possible, as 
	$(\sqrt{2}+\sqrt{3})+(\sqrt{2}-\sqrt{3})=2\sqrt{2}\not\in\Q$, 
	$(\sqrt{2}+\sqrt{3})+(-\sqrt{2}+\sqrt{3})=2\sqrt{3}\not\in\Q$ and 
	$(\sqrt{2}+\sqrt{3})(-\sqrt{2}-\sqrt{3})=-5-2\sqrt{6}\not\in\Q$.
\end{example}

\begin{proposition}
	Let $F/K$ be a subextension and $E/K$. Then
	\[
	[E:K]=[E:F][F:K].
	\]
\end{proposition}

\begin{proof}
	Let $\{e_i:i\in I\}$ be a basis of $E$ over $K$
	and $\{f_j:j\in J\}$ be a basis of $F$ over $K$. If $x\in E$,
	then $x=\sum_i \lambda_ie_i$ (finite sum) 
	for some $\lambda_i\in F$. For each $i$, 
	$\lambda_i=\sum_j a_{ij}f_j$ (finite sum)
	for some $a_{ij}\in K$. Then 
	$x=\sum_i\sum_j a_{ij}(f_je_i)$. This means
	that $\{f_je_i:i\in I,j\in J\}$ generates
	$E$ as a $K$-vector space. Let us prove that 
	$\{f_je_i:i\in I,j\in J\}$
	is linearly independent. If $\sum_i\sum_j a_{ij}(f_je_i)=0$ (finite sum)
	for some $a_{ij}\in K$, 
	then
	\begin{align*}
		0=\sum_i\left(\sum_j a_{ij}f_j\right)e_i&\implies
		\sum_j a_{ij}f_j=0\text{ for all $i\in I$}\\
		&\implies 
		a_{ij}=0\text{ for all $i\in I$ and $j\in J$}.\qedhere
	\end{align*}
\end{proof}

We state a lemma:

\begin{lemma}
If $A$ is a finite-dimensional commutative algebra over $K$ 
and $A$ is an integral domain, then $A$ is a field. 
\end{lemma}

\begin{proof}
	Let $a\in A\setminus\{0\}$. We need to prove that there exists $b\in A$
	such that $ab=1$. Let $\theta\colon A\to A$, $x\mapsto ax$. Clearly
	$\theta$ is an algebra homomorphism. It is injective, since $A$ is an
	integral domain.  Since $\dim_KA<\infty$, it follows that $\theta$ is an
	isomorphism. In particular, $\theta(A)=A$, which means that there exists
	$b\in A$ such that $1=ab$. 
\end{proof}

\begin{theorem}
	Let $E/K$ be an extension and $x\in E\setminus K$.
	The following statements are equivalent:
	\begin{enumerate}
		\item $x$ is algebraic over $K$.
		\item $\dim_KK[x]<\infty$.
		\item $K[x]$ is a field.
		\item $K[x]=K(x)$. 
	\end{enumerate}
\end{theorem}

\begin{proof}
	We first prove $1)\implies 2)$. Let $z\in K[x]$, say $z=h(x)$ for some $h\in K[X]$. There exists
	$g\in K[X]$ such that $g\ne 0$ and $g(x)=0$. Divide $h$ by $g$ to obtain 
	polynomials $q,r\in K[X]$ such that $h=gq+r$, where $r=0$ or $\deg r<\deg g$. This implies that
	\[
		z=h(x)=g(x)q(x)+r(x)=r(x).
	\]
	If $\deg g=m$, then $r=\sum_{i=0}^{m-1}a_iX^i$ for some $a_0,\dots,a_{m-1}\in K$. Thus
	$z=\sum_{i=0}^{m-1}a_ix^i$, so $K[x]\subseteq\langle 1,x,\dots,x^{m-1}\rangle$. 

	The previous lemma proves that $2)\implies 3)$. 

	It is trivial that $3)\implies 4)$. 

	It remains to prove that $4)\implies 1)$. Let us prove that $K(x)\subseteq K[x]$. 
	Since $x\ne 0$, $1/x\in K[x]$. There exists $a_0,\dots,a_n\in K$ such that
	$1/x=a_0+a_1x+\cdots+a_nx^n$. Thus
	\[
		a_nx^{n+1}+\cdots+a_1x^2+a_0x-1\ne 0
	\]
	and $x$ is a root of $a_nX^{n+1}+\cdots+a_0X-1\in K[X]$. 
\end{proof}

Note that if $x$ is algebraic over $K$, then
$K[x]\simeq K[X]/(f(x,K))$. 

\begin{corollary}
	If $E/K$ is finite, then $E/K$ is algebraic. 
\end{corollary}

\begin{proof}
	Let $n=[E:K]$ and $x\in E$. The set $\{1,x,\dots,x^n\}$ is linearly dependent, 
	so there exist $a_0,\dots,a_n\in K$ not all zero such that
	$a_0+a_1x+\cdots+a_nx^n=0$. Thus $x$ is a root of the non-zero
	polynomial $a_0+a_1X+\cdots+a_nX^n\in K[X]$. 
\end{proof}

We note that the converse of the previous corollary does not hold. 

\begin{corollary}
	If $E/K$ is an extension and $x_1,\dots,x_n\in E$ 
	are algebraic over $K$, then 
	$K(x_1,\dots,x_n)/K$ is finite and
	$K(x_1,\dots,x_m)=K[x_1,\dots,x_n]$. 
\end{corollary}

\begin{proof}
	We proceed by induction on $n$. The case $n=1$ follows immediately from 
	the theorem. So assume the result holds for some $n\geq1$. Since the extensions 
	$K(x_1,\dots,x_n)/K(x_1,\dots,x_{n-1})$ and $K(x_1,\dots,x_{n-1})/K$ are
	both finite, it follows that $K(x_1,\dots,x_n)/K$ is finite. Moreover, 
	\begin{align*}
	K(x_1,\dots,x_n)&=K(x_1,\dots,x_{n-1})(x_n)\\
	&=K(x_1,\dots,x_{n-1})[x_n]=K[x_1,\dots,x_{n-1}][x_n]=K[x_1,\dots,x_n].\qedhere
    \end{align*}
\end{proof}

\begin{corollary}
	Let $E=K(S)$. Then $E/K$ is algebraic if and only if
	$x$ is algebraic over $K$ for all $x\in S$. 
\end{corollary}

\begin{proof}
	Let us prove the non-trivial implication. Let $z\in K(S)$. In particular, 
	there exists a finite subset $T\subseteq S$ such that 
	$z\in K(T)$. The previous corollary implies that $K(T)/K$ is algebraic and
	hence $z$ is algebraic. 
\end{proof}

\begin{corollary}
	If $E/K$ is  an extension, then $\overline{K}_E$ 
	is a subfield of $E$ that contains $K$. Moreover, 
	$K(\overline{K}_E)/K$ is algebraic. 
\end{corollary}	

\begin{proof}
    By definition, $K(\overline{K}_E)/K$ is algebraic. 
    Thus $K(\overline{K}_E)\subseteq\overline{K}_E$. From this it follows that
    $K(\overline{K}_E)=\overline{K}_E$. 
\end{proof}

The following exercise is now almost trivial:

\begin{exercise}
    Let $E/K$ be an extension. Prove that $E/K$ is algebraic if and only if $E/K$ 
    is finite of finite type. 
\end{exercise}


%\begin{theorem}[Galois]
%	\index{Galois' theorem}
%	For every prime number $p$ and every $m\geq1$
%	there exists a field of size $p^m$. 
%\end{theorem}
%
%\begin{proof}
%\end{proof}
%
%
Algebraic field extensions form a nice class of extensions. The same happens
with finite field extensions. 

\begin{proposition}
	Let $F/K$ is a subextension of $E/K$. Then $E/K$ is algebraic 
	if and only if $E/F$ and $F/K$ are algebraic. 
\end{proposition}

\begin{proof}
    We know that if $E/K$ is algebraic, then $E/F$ and $F/K$ are both algebraic. 
    Let us assume that $E/F$ and $F/K$ are both algebraic. Let $x\in E$ and 
    let $L$ be the subextension over $K$ generated by the coefficients of $f(x,F)$, 
    the minimal polynomial of $x$ over $F$. Then $L/K$ is finite, since it is generated
    by finitely many algebraic elements. Moreover, $x$ is algebraic over $L$. Since 
    \[
    [L(x):K]=[L(x):L][L:K]<\infty,
    \]
    $L(x)/K$ is algebraic. In particular, $x$ is algebraic over $K$. 
\end{proof}

\begin{exercise}
	Let $F/K$ is a subextension of $E/K$. Prove that $E/K$ is finite 
	if and only if $E/F$ and $F/K$ are finite. 
\end{exercise}

\begin{exercise}
	Let $E/K$ and $F/K$ be extensions, where both $E$ and $F$ are subfields of 
	a field $L$. If $F/K$ is algebraic, then $EF/E$ is algebraic.
\end{exercise}

% \begin{proof}
%     If $F/K$ is algebraic, then $EF/E=E(F)/E$ is algebraic, as it is generated by 
%     algebraic elements over $E$.  
% \end{proof}

\begin{exercise}
	Let $E/K$ and $F/K$ be extensions, where both $E$ and $F$ are subfields of 
	a field $L$. If $F/K$ is finite, then $EF/E$ is finite.
\end{exercise}

The solution to the previous exercise shows, in particular, that $[EF:E]\leq [F:K]$. 



%\begin{theorem}[Galois]
%	\index{Galois' theorem}
%	For every prime number $p$ and every $m\geq1$
%	there exists a field of size $p^m$. 
%\end{theorem}
%
%\begin{proof}
%\end{proof}
%
%
% Algebraic field extensions form a nice class of extensions. The same happens
% with finite field extensions. 

% \begin{proposition}
% 	Let $F/K$ is a subextension of $E/K$. Then $E/K$ is algebraic (resp. finite)
% 	if and only if $E/F$ and $F/K$ are algebraic (resp. finite). 
% \end{proposition}

% \begin{proof}
% 	From the formula 
% 	$[E:K]=[E:F][F:K]$ it follows that 
% 	$E/K$ is finite if and only if $E/F$ and $F/K$ are
% 	both finite. 

% 	If $E/K$ is algebraic, then $E/F$ and $F/K$ are both algebraic. Conversely,
% 	suppose now that both $E/F$ and $F/K$ are algebraic. For $x\in E$ let $L$
% 	be the extension of $K$ generated by the coefficients of $f(x,F)$, the
% 	minimal polynomial of $x$ over $F$. Then $L$ is finite, as it is generated
% 	by finitely many algebraic elements. Moreover, $x$ is algebraic over $L$.
% 	Since $[L(x):K]=[L(x):L][L:K]<\infty$, $L(x)/K$ is algebraic. In
% 	particular, $x$ is algebraic over $K$. 
% \end{proof}

% \begin{proposition}
% 	Let $E/K$ and $F/K$ be extensions, where both $E$ and $F$ are subfields of
% 	a field $L$. If $F/K$ is algebraic (resp. finite), then $EF/E$ is algebraic
% 	(resp. finite).
% \end{proposition}

% \begin{proof}
% 	Now we prove that if $F/K$ is finite, then $EF/E$ is finite. For that purpose,
% 	we show that $[EF:E]<[F:K]$. Recall that $EF=E(F)$. The elements of $F$ are
% 	algebraic over $K$, so they are algebraic over $E$. In particular, $E(F)/E$ is algebraic
% 	and $E(F)=E[F]$. Let $z\in EF$, say $z=\sum_i x_it_i$ for some $x_i\in E$ and $t_i\in F$. 
% 	The extension $F/K$ is finite, so let $\{f_1,\dots,f_m\}$ be a basis of $F$ over $K$. Then
% 	each $t_i$ can be written as $t_i=\sum_ja_{ij}f_j$ for some $a_{ij}\in K$. Then
% 	\[
% 		z=\sum_j\left(\sum_i a_{ij}x_i\right)f_j
% 	\]
% 	and thus $\{f_1,\dots,f_m\}$ generates $EF$ as a vector space over $E$. 
% \end{proof}

% \[\begin{tikzcd}
% 	& EF \\
% 	E && F \\
% 	& K
% 	\arrow[no head, from=1-2, to=2-1]
% 	\arrow[no head, from=2-1, to=3-2]
% 	\arrow[no head, from=3-2, to=2-3]
% 	\arrow[no head, from=1-2, to=2-3]
% \end{tikzcd}\]

\begin{lemma}
	Let $\sigma\colon K\to L$ be a field homomorphism. Then there exists an extension
	$E/K$ and a field isomorphism $\varphi\colon E\to L$
	such that $\varphi|_K=\sigma$. 
\end{lemma}

\begin{proof}
	Let $A$ be a set in bijection with $L\setminus\sigma(K)$ and disjoint with $K$. 
	Let $E=K\cup A$. If $\theta\colon A\to L\setminus\sigma(K)$ is bijective, then 
	let 
	\[
		\varphi\colon E\to L,
		\quad
		\varphi(x)=\begin{cases}
			\sigma(x) & \text{if $x\in K$},\\
			\theta(x) & \text{if $x\in A$}.
		\end{cases}
	\]
	Then $\varphi$ is a bijective map such that $\varphi|_K=\sigma$. 
	Transport the operations of $L$ onto $E$, that is 
	to define binary operations on $E$ as follows: 
	\begin{align*}
		&(x,y)\mapsto x\oplus y=\varphi^{-1}(\varphi(x)+\varphi(y)), && 
		(x,y)\mapsto x\odot y=\varphi^{-1}(\varphi(x)\varphi(y)).
	\end{align*}
	Then, for example, 
	\[
		x\oplus y=\varphi^{-1}(\varphi(x)+\varphi(y))=\varphi^{-1}(\sigma(x)+\sigma(y))
		=\varphi^{-1}(\sigma(x+y))=\varphi^{-1}(\varphi(x+y))=x+y
	\]
	for all $x,y\in K$. 
\end{proof}

If $\sigma\colon A\to B$ is a ring homomorphism, then $\sigma$ induces a ring
homomorphism $\overline{\sigma}\colon A[X]\to B[X]$,
$\sum_ia_iX^i\mapsto\sum\sigma(a_i)X^i$. 

\begin{theorem}
	Let $K$ be a field and $f\in K[X]$ be such that $\deg f>0$. Then 
	there exists an extension $E/K$ such that $f$ admits a root in $E$. 
\end{theorem}

\begin{proof}
	We may assume that $f$ is irreducible over $K$. Let $L=K[X]/(f)$ and 
	$\pi\colon K[X]\to L$ be the canonical map. Then $L$ 
	is a field. The field homomorphism $\sigma\colon K\to L$, $a\mapsto \pi(aX^0)$. 
	Let $g=\overline{\sigma}(f)\in L[X]$. 

	We claim that $\pi(X)$ is a root of $g$ in $L$. Suppose that $f=\sum_i a_iX^i$. 
	Then 
	\begin{align*}
		g(\pi(X))&=\overline{\sigma}(f)(\pi(X))\\
		&=\sum_i \sigma(a_i)\pi(X)^i
		=\sum_i\pi(a_iX^0)\pi(X^i)=\pi(\sum a_iX^i)=\pi(f)=0.
	\end{align*}
	The previous lemma states that 
	there exists an extension $E/K$ and an isomorphism $\varphi\colon E\to L$
	such that $\varphi|_K=\sigma$. If $u=\pi(X)$, then $\varphi^{-1}(u)$ is a root of $f$ in $E$, 
	as 
	\begin{align*}
		\varphi(f(\varphi^{-1}(u)))&=\varphi\left(\sum_ia_i\varphi^{-1}(u)^i\right)
		=\varphi\left(\sum_ia_i\varphi^{-1}(u^i)\right)\\
		&=\sum_i\varphi(a_i)u^i=\sum_i\sigma(a_i)u^i=g(u)=0.\qedhere
	\end{align*}
\end{proof}

As a corollary, if $K$ is a field and $f_1,\dots,f_n\in K[X]$ are polynomials 
of positive degree, then there exists an extension $E/K$  such that 
each $f_i$ admits a root in $E$. This is proved by induction on $n$.  

\begin{definition}
	A field $K$ is \textbf{algebraically closed} if each $f\in K[X]$ 
	of positive degree admits a root in $K$. 
\end{definition}

The \emph{fundamental theorem of algebra} states that $\C$ is algebraically closed. A
typical proof uses complex analysis.  Later we will give a proof of this result
using Galois theory. 

\begin{proposition}
	The following statements are equivalent:
	\begin{enumerate}
		\item $K$ is algebraically closed.
		\item If $f\in K[X]$ is irreducible, then $\deg f=1$.
		\item If $f\in K[X]$ is non-zero, then $f$ decomposes linearly in $K[X]$, that is
			\[
				f=a\prod_{i=1}^n(X-\alpha_i)^{m_i}
			\]
			for some $a\in K$ and $\alpha_1,\dots,\alpha_n\in K$. 
		\item If $E/K$ is algebraic, then $E=K$. 
	\end{enumerate}
\end{proposition}

\begin{proof}
	$1)\implies 2\implies 3)$ are exercises.  
	
	Let us prove that $3)\implies
	4)$. Let $x\in E$. Decompose $f(x,K)$ linearly in $K[X]$ as
	$f(x,K)=a\prod_{i=1}^n(X-\alpha_i)$ and evaluate on $x$ to obtain that
	$x=\alpha_j$ for some $j$. 
	
	To prove that $4)\implies 1)$ let $f\in K[X]$ be
	such that $\deg f>0$. There exists an extension $E/K$ such that $f$ has a
	root $x$ in $E$. The extension $K(x)/K$ is algebraic and hence $K(x)=K$, so
	$x\in K$. 
\end{proof}



\topic{Artin's theorem}

\begin{definition}
	The \textbf{algebraic closure} of a field $K$ is an algebraic extension $C/K$ 
	such that $C$ is algebraically closed. 
\end{definition}

For example, $\C/\R$ is an algebraic closure but $\C/\Q$ it is not. 

\begin{proposition}
	Let $C$ be algebraically closed and $\sigma\colon K\to C$ be a field homomorphism. If $E/K$ 
	is algebraic, then there exists a field homomorphism 
	$\varphi\colon E\to C$ such that 
	$\varphi|_K=\sigma$. 
\end{proposition}

\begin{proof}
	Suppose first that $E=K(x)$ and let $f=f(x,K)$. Let $\overline{\sigma}(f)\in C[X]$ 
	and let $y\in C$ be a root of $\overline{\sigma}(f)$. If $z\in E$, then $z=g(x)$ for
	some $g\in K[X]$. Let $\varphi\colon E\to C$, $z\mapsto \overline{\sigma}(g)(y)$. 

	The map $\varphi$ is well-defined. If $z=h(x)$ for some $h\in K[X]$, then
	\[
	0=g(x)-h(x)=(g-h)(x)
	\]
	and thus $f$ divides $g-h$. In particular, $\overline{\sigma}(f)$ divides
    $\overline{\sigma}(g-h)=\overline{\sigma}(g)-\overline{\sigma}(h)$ and hence
    $(\overline{\sigma}(g)-\overline{\sigma}(h))(y)=0$. 

	It is an exercise to show that the map $\varphi$ is a ring homomorphism.
	
	Let $a\in K$. Since $a=(aX^0)(x)$, it follows that $\varphi|_K=\sigma$, as 
	\[
	\varphi(a)=\overline{\sigma}(aX^0)(y)=(\sigma(a)X^0)(y)=\sigma(a)
	\]
	and 
	$\varphi(x)=\overline{\sigma}(X)(y)=y$. 
	
	Let us now prove the proposition in full generality. Let 
	$X$ be the set of pairs $(F,\tau)$, where $F$ is a subfield of $E$ that contains $K$ and
	$\tau\colon F\to C$ is a field homomorphism such that $\tau|_K=\sigma$. Note that
	$(K,\sigma)\in X$, so $X$ is non-empty. Moreover, $X$ is partially ordered by
	\[
	(F,\tau)\leq (F_1,\tau_1)\Longleftrightarrow F\subseteq F_1\text{ and }\tau_1|_F=\tau.
	\]
	If $\{(F_i,\tau_i):i\in I\}$ is a chain in $X$, then $F=\cup_{i\in I}F_i$ is a subfield of $E$
	that contains $K$. Moreover, if $z\in F$, then $z\in F_i$ for some $i\in I$ and 
	then one defines $\tau(z)=\tau_i(z)$. It is an exercise to prove that $\tau$ is well-defined.
	Since $(F,\tau)\in X$ is an upper bound, Zorn's lemma implies that there exists
	a maximal element 
	$(E_1,\theta)\in X$. We claim that $E=E_1$. If not, let $z\in E\setminus E_1$. 
	Since we know the proposition is true for the extension $E_1(z)/K$, 
	let  
	$\rho\colon E_1(z)\to C$ be a field homomorphism such that $\rho|_{E_1}=\sigma$. Then, in particular, 
	$\rho|_K=\sigma$. This implies that $(E_1(z),\rho)\in X$ and hence
	$(E_1,\theta)<(E_1(z),\rho)$, a contradiction to the maximality of $(E_1,\theta)$. 
\end{proof}

The previous proposition will be used to prove 
that the algebraic closure always exists. 

\begin{theorem}[Artin]
	\index{Artin's theorem}
	Let $K$ be a field. Then $K$ admits an algebraic closure $C/K$. If $C_1/K$
	is an algebraic closure, then the extensions $C/K$ and $C_1/K$ are
	isomorphic. 
\end{theorem}

\begin{proof}
    Let us first prove the uniqueness. The previous proposition implies the existence of 
    an extensions homomorphism $\varphi\colon C\to C_1$. Let $y\in C_1$ and $f=f(y,K)$ be 
    the minimal polynomial of $y$ in $K$. Since $f$ admits a factorization
    \[
        f=\lambda\prod (X-\alpha_i)^{m_i}
    \]
    in $C[X]$, it follows that
    \[
    f=\overline{\varphi}(f)=\prod (X-\varphi(\alpha_i))^{m_i}
    \]
    Since $0=f(y)$, we conclude that $y=\varphi(\alpha_j)$ for some $j$. In particular, $\varphi$ is
    surjective and hence $\varphi$ is bijective. 
    
    We now prove the existence. Let us assume that $K$ admits an extension $E/K$ 
    with $E$ algebraically closed. Let $F=...$. Then $F/K$ is algebraic. Let $g\in F[X]$ be such that
    $\deg g>0$. Since $E$ is algebraically closed, $g$ admits a root $\alpha$ in $E$. In particular, $\alpha$
    is algebraic over $F$ and hence $\alpha$ is algebraic over $K$. This implies that $\alpha\in F$, thus
    $F$ is algebraically closed. This proves that $F/K$ is an algebraic closure. 
    
    Let us prove that there exists an extension $E_1/K$ such that
    every polynomial $f\in K[X]$ with $\deg f>0$ has a root in $E_1$. Let 
    $\{f_i:i\in I\}$ be the family of monic irreducible polynomials with coefficients in $K$. 
    We may think that $f_i=f_i(X_i)$. 
    Let $R=K[\{X_i:i\in I\}]$ and let $J$ be the ideal of $R$ 
    generated by the $f_i(X_i)$. We claim that $J\ne R$. If not, $1\in J$, so
    \[
    1=\sum_{i=1}^m g_jf_{i_j}(X_j)
    \]
    for some $g_1,\dots,g_m\in R$. There exists an extension $F/K$ such that
    $f_{i_j}$ has a root $\alpha_j$ in $F$ for all $j$. Let 
    \[
    \sigma\colon R\to F,\quad
    \sigma(X_k)=\begin{cases}
        \alpha_j & \text{if $k=i_j$},\\
        0 & \text{if $k\not\in\{i_1,\dots,i_m\}$}.
        \end{cases}
    \]
    Then $1=\sigma(1)=\sum_{j=1}^m\sigma(g_j)f_{i_j}(\alpha_j)$, a contradiction. 
    
    Since $J$ is a proper ideal, it is contained in a maximal ideal $M$. Let $L=R/M$ 
    and let $\sigma\colon K\to L$ be given by...
    Then $\pi(X_i)$ is a root of $\overline{\sigma}(f_i)$ for all $i$ 
    and there exists an extension $E_1/K$ such that
    every $f_i$ has a root in $E_1$. Proceeding in this way, we construct
    a sequence
    \[
    E_1\subseteq E_2\subseteq\cdots
    \]
    of fields such that every polynomial of positive degree and coefficients in $E_k$ 
    admits a root in $E_{k+1}$. Let $E=\cup E_k$. We claim that $E$ is algebraically closed. In fact, 
    let $g\in E[X]$ be such that $\deg g>0$. Then, since $g\in E_r[X]$ for some $r$, it follows
    that $g$ has a root in $E_{r+1}\subseteq E$. 
\end{proof}

\topic{Decomposition fields}

\begin{definition}
	Let $K$ be a field and $f\in K[X]$ be such that $\deg f>0$. A \textbf{decomposition field}
	of $f$ over $K$ is field $E$ that contains $K$ and that satisfies the following properties:
	\begin{enumerate}
		\item $f$ factorizes linearly in $E[X]$. 
		\item if $F$ is a field such that $K\subseteq F\subseteq E$ and 
			$f$ factorizes
			linearly in $F[X]$, then $F=E$. 
	\end{enumerate}
\end{definition}

Easy examples: 

\begin{example}
	$\C$ is a decomposition field of $X^2+1\in\R[X]$. 
\end{example}

\begin{example}
	$\Q[\sqrt{2}]$ is a decomposition field of $X^2-2\in\Q[X]$. 
\end{example}

\begin{example}
	$\Q(\sqrt[3]{2})$ is not a decomposition field of $X^3-2\in\Q[X]$. However, if
	$\omega$ ia a primitive cubic root of one, then 
	$\Q(\sqrt[3]{2},\omega)$ is is a decomposition field of $X^3-2\in\Q[X]$. 
\end{example}

\begin{proposition}
	$E$ is a decomposition field of $f\in K[X]$ if and only if
	$f$ factorizes linearly in $E[X]$ and $E=K(x_1,\dots,x_n)$ where 
	$x_1,\dots,x_n$ are roots of $f$. 
\end{proposition}

\begin{proof}
\end{proof}



\bibliographystyle{abbrv}
\bibliography{refs}

\printindex

\end{document}
