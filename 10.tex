\section{Lecture -- Week 10}


\begin{corollary}
    Let $a,b,c\in \Z$ be such that $a^2+b^2=c^2$. Then 
    \[
    (a,b,c)=\lambda(r^2-s^2,-2rs, r^2+s^2)
    \]
    for some $r,s\in\Z$ and some $\lambda\in\Z$.
\end{corollary}

\begin{proof}
    We work with the extension $\Q(i)/\Q$. Note that 
    $\Gal(\Q(i),\Q)=\{\id,\gamma\}$ is cyclic, where 
    $\gamma\colon\Q(i)\to\Q(i)$, $z\mapsto\overline{z}$, is the complex conjugation. 
    We may assume that $c\ne 0$, otherwise $a=b=0$ and the result is trivial.  
    Write $(a/c)^2+(b/c)^2=1$ and let 
    \[
    \alpha=(a/c)+(b/c)i\in\Q(i).
    \]
    Then
    $\norm_{\Q(i)/\Q}(\alpha)=1$. 
    By Hilbert's theorem, 
    there exists $\beta\in\Q(i)\setminus\{0\}$ such that 
    \[
    \alpha=a+bi=\frac{\gamma(\beta)}{\beta}.
    \]
    Note that if $m\in\Z\setminus\{0\}$, then 
    $\frac{\gamma(m\beta)}{m\beta}=\frac{\gamma(\beta)}{\beta}$. 
    There exists $m\in\Z\setminus\{0\}$ such that 
    $m\beta\in\Z[i]$, say $m\beta=r+is$ with $r,s\in\Z$. Then
    \[
    \alpha=\frac{\gamma(\beta)}{\beta}=\frac{\gamma(m\beta)}{m\beta}=
    \frac{r-is}{r+is}=\frac{r^2-s^2-2rsi}{r^2+s^2}.
    \]
    From this the claim follows. 
\end{proof}

\begin{exercise}
% https://abel.math.harvard.edu/~elkies/Misc/hilbert.pdf
Let $A,B\in\Z$ be such that $A^2-4B$ is not a square. Prove that 
a solution $(x,y,z)\in\Z^3$ of $x^2 + Axy + By^2 = z^2$
is proportional to 
\[
(r^2-Bs^2,2rs+As^2,r^2+Ars+Bs^2).
\]
\end{exercise}


%\begin{proof}
%    
%    By the previous theorem, $H^1(G,M)$ is a
%\end{proof}
%
% \begin{definition}
%     \index{Derivation}
%     Let $A$ be a $G$-module. 
%     A \emph{derivation} of $A$ is a map $d\colon G\to A$ such that
%     $d(gh)=g\cdot d(h)+d(g)$ for all $g,h\in G$. 
% \end{definition}

% \index{Inner derivation}
% Let $A$ be a $G$-module and $a\in A$. The map 
% $d(g)=g\cdot a-a$ is a derivation of $A$. 
% Such derivations as known as \emph{inner derivations}. 

% \begin{exercise}
%     Let $E/K$ be a finite Galois extension and $G=\Gal(E/K)$. 
%     Then the (multiplicative) group $E^\times$ is a $G$-module with
%     $\sigma\cdot x=\sigma(x)$. 
% \end{exercise}

% In the context of the previous exercise, 
% let $d\colon G\to E^\times$ be a derivation. For $\tau\in G$ let 
% $x_\tau=d(\tau)$. Then
% \[
% x_{\sigma\tau}=d(\sigma\tau)=(\sigma\cdot d(\tau))d(\sigma)=\sigma(x_\tau)x_\sigma.
% \]
% Is this true that $x_\sigma=\frac{\sigma(c)}{c}=(\sigma\cdot c)c^{-1}$ 
% for some $c\in E^\times$?

% \begin{proposition}
%     Let $E/K$ be a finite Galois extension and $G=\Gal(E/K)$. 
%     \begin{enumerate}
%         \item Let $\{x_\tau:\tau\in G\}\subseteq E^\times$ be such that 
%             $x_{\sigma\tau}=\sigma(x_\tau)x_\sigma$. Then there exists
%             $c\in E^\times$ such that $x_\sigma=\frac{\sigma(c)}{c}$ for all $\sigma\in G$. 
%         \item Let $\{x_\tau:\tau\in G\}\subseteq E$ be such that 
%             $x_{\sigma\tau}=\sigma(x_\tau)+x_\sigma$. Then there exists
%             $c\in E$ such that $x_\sigma=\sigma(c)-c$ for all $\sigma\in G$.  
%     \end{enumerate}
% \end{proposition}

% \begin{proof}
%     We prove 1). By Dedekind's theorem, the homomorphism 
%     $\sum_{\tau\in G}x_{\tau}^{-1}\tau$ is non-zero. Thus there exists
%     $a\in E^{\times}$ such that $\sum_{\tau\in G}x_{\tau}^{-1}\tau(a)=c\in E^{\times}$. 
%     If $\sigma\in G$, 
%     then 
%     \[
%     \sigma(c)=\sum_{\tau\in G}\sigma(x_{\tau}^{-1})(\sigma\tau)(a)=
%     \sum_{\tau\in G}x_{\sigma\tau}^{-1}x_{\sigma}\sigma\tau(a)
%     =x_{\sigma}\sum_{\tau\in G}x_{\sigma\tau}^{-1}\sigma\tau(a)=x_{\sigma}c.
%     \]
    
%     We now prove 2). By Dedekind's theorem, $\sum_{\tau\in G}\tau\ne0$. Since
%     it is a non-zero 
%     linear form in $E$, it is surjective. In particular, there exists 
%     $a\in E$ such that $\sum_{\tau\in G}\tau(a)=1$. Let $c=-\sum_{\tau\in G}x_\tau\tau(a)$. 
%     If $\sigma\in G$, then 
%     \begin{align*}
%         \sigma(c) &= -\sum_{\tau\in G}\sigma(x_{\tau})\sigma\tau(a)\\
%         &=-\sum_{\tau\in G}(x_{\sigma\tau}-x_{\sigma})\sigma\tau(a)\\
%         &=-\sum_{\tau\in G}x_{\sigma\tau}\sigma\tau(a)+\sum_{\tau\in G}x_{\sigma}\sigma\tau(a)
%         =c+x_{\sigma}.
%     \end{align*}
%     Hence $x_{\sigma}=\sigma(c)-c$. 
% \end{proof}

% The following result is known as Hilbert's 90 theorem. 

% \begin{theorem}[Hilbert's theorem 90]
% \index{Hilbert's theorem 90}
%     Let $E/K$ be a cyclic extension of degree $n$ with Galois group $G=\Gal(E/K)=\langle\sigma\rangle$. 
%     \begin{enumerate}
%         \item Let $x\in E^{\times}$ be such that $\norm_{E/K}(x)=1$, 
%         There exists $b\in E^{\times}$ such that $x=\sigma(b)/b$. 
%         \item Let  $x\in E$ be such that $\trace_{E/K}(x)=0$. 
%         There exists $b\in E$ such that $x=\sigma(b)-b$. 
%     \end{enumerate}
% \end{theorem}

% \begin{proof}
%     Let us prove 1). Note that $G=\{\sigma^i:0\leq i<n\}$. For $i\in\{0,\dots,n-1\}$ let 
%     $x_{\sigma^i}=\prod_{k=0}^{i-1}\sigma^k(x)$. In particular, $x_{\sigma}=x$. We now
%     check that $\{x_{\sigma^i}\}$ satisfy the assumptions of the previous proposition:
%     \begin{align*}
%         \sigma^j(x_{\sigma^i}) &= \prod_{k=0}^{i-1}\sigma^{k+j}(x)
%         =\prod_{k=j}^{i+j-1}\sigma^k(x)\\
%         &=\prod_{k=0}^{i+j-1}\sigma^k(x)\left(\prod_{k=0}^{j-1}\sigma^k(x)\right)^{-1}
%         =\prod_{k=0}^{i+j-1}\sigma^k(x)x_{\sigma^j}^{-1}.
%     \end{align*}
%     If $i+j<n$, then 
%     \[
%     \sigma^j\sigma^i=\sigma^{i+j}
%     \implies
%     \sigma^j(x_{\sigma^i})=x_{\sigma^i\sigma^j}x_{\sigma^j}^{-1}
%     \implies
%     x_{\sigma^j\sigma^i}=\sigma^j(x_{\sigma^i})x_{\sigma^j}.
%     \]
%     If $i+j=n+r$ for some $r\in\{0,\dots,n-1\}$, then, since $i+j<2n$ and
%     $\sigma^i\sigma^j=\sigma^r$, 
%     \[
%     \sigma^j(x_{\sigma^i})=\prod_{k=0}^{n-1}\sigma^k(x)\prod_{k=n}^{n+r-1}\sigma^k(x)x_{\sigma^j}^{-1}
%     =\prod_{k=0}^{r-1}\sigma^k(x)x_{\sigma^j}^{-1}
%     =x_{\sigma^r}x_{\sigma^j}^{-1}=x_{\sigma^j\sigma^i}x_{\sigma^j}^{-1}
%     \]
%     and hence $x_{\sigma^j\sigma^i}=\sigma^j(x_{\sigma^i})x_{\sigma^j}$. By 
%     the previous lemma, there exists $c\in E^{\times}$ such that 
%     $x_{\sigma^i}=\frac{\sigma^i(c)}{c}$. In particular, $x=x_{\sigma}=\sigma(c)/c$. 
    
%     The second statement is similar and it is left as an exercise. 
% \end{proof}

% If $A$, $B$ and $C$ are groups (written multiplicatively) 
% and $f\colon A\to B$ and $g\colon B\to C$ are group homomorphism, 
% the sequence 
% \[\begin{tikzcd}
% 	A & B & C 
% 	\arrow["f", from=1-1, to=1-2]
% 	\arrow["g", from=1-2, to=1-3]
% \end{tikzcd}\]
% of groups and homomorphisms 
% is said to be \emph{exact} if $f(A)=\ker g$. 
% For example, the sequence 
% \[\begin{tikzcd}
% 	1 & B & C 
% 	\arrow[from=1-1, to=1-2]
% 	\arrow["g", from=1-2, to=1-3]
% \end{tikzcd}\]
% is exact if and only if $g$ is injective, as the first map represents the trivial homomorphism. 
% Similarly, the sequence 
% \[\begin{tikzcd}
% 	A & B & 1
% 	\arrow["f", from=1-1, to=1-2]
% 	\arrow[from=1-2, to=1-3]
% \end{tikzcd}\]
% is exact if and only if $f$ is surjective. 

% \begin{corollary}
% Let $E/K$ be finite and cyclic with $\Gal(E/K)=\langle\sigma\rangle$. 
% \begin{enumerate}
%     \item The sequence 
%     \[\begin{tikzcd}
% 	1 & {K^{\times}} & {E^{\times}} & {E^{\times}} & {K^{\times}}
% 	\arrow[from=1-1, to=1-2]
% 	\arrow[hook, from=1-2, to=1-3]
% 	\arrow["\rho", from=1-3, to=1-4]
% 	\arrow["{\norm_{E/K}}", from=1-4, to=1-5]
%     \end{tikzcd}\]
%     is exact, where $\rho(z)=\sigma(z)/z$. 
%      \item The sequence 
%     \[\begin{tikzcd}
% 	0 & {K} & {E} & {E} & {K}
% 	\arrow[from=1-1, to=1-2]
% 	\arrow[hook, from=1-2, to=1-3]
% 	\arrow["\lambda", from=1-3, to=1-4]
% 	\arrow["{\trace{E/K}}", from=1-4, to=1-5]
%     \end{tikzcd}\]
%     is exact, where $\lambda(z)=\sigma(z)-z$. 
% \end{enumerate}
% \end{corollary}

% \begin{proof}
%     We only prove 1). Note that 
%     $K^{\times}\hookrightarrow E^{\times}$ is the inclusion map. If 
%     $z\in\ker\rho$, then $\sigma(z)=z$ and hence $z\in K$. The 
%     sequence 
%     \[\begin{tikzcd}
% 	{E^{\times}} & {E^{\times}} & {K^{\times}}
% 	\arrow["\rho", from=1-1, to=1-2]
% 	\arrow["{\norm_{E/K}}", from=1-2, to=1-3]
%     \end{tikzcd}\]
%     is exact by Hilbert's theorem 90. 
% \end{proof}

% % 107/136

\begin{proposition}
    Let $n\geq2$ and $K$ be a field containing a primitive $n$-root of one. If 
    $a\in K^{\times}$ and 
    $E/K$ is a splitting field of $f=X^n-a$, then $E/K$ is cyclic
    of degree $d$, where $d$ divides $n$. Moreover, 
    \[
    d=\min\{k:a^k\in K^n\},
    \]
    where $K^n=\{x\in K:x=y^n\text{ for some $y\in K$}\}$. 
    Conversely, 
    if $E/K$ is cyclic of degree $n$, then $E/K$ is a decomposition
    field of an irreducible polynomial 
    of the form $X^n-a$ for some $a\in K^{\times}$. 
\end{proposition}

\begin{proof}
    A splitting field of $f$ over $K$ is
    of the form $K(\alpha)$, where $\alpha^n=a$. Thus 
    $K(\alpha)/K$ is a Galois extension. If $\sigma\in\Gal(K(\alpha)/K)$, 
    then $\sigma(\alpha)$ is a root of $f$, so 
    $\sigma(\alpha)=\omega_\sigma\alpha$, where
    $\omega_\sigma\in G_n(K)$. This means that 
    there exists an injective map
    \[
    \lambda\colon\Gal(K(\alpha)/K)\to G_n(K),
    \quad
    \sigma\mapsto\omega_{\sigma}.
    \]
    Moreover, $\lambda$ is a group homomorphism, as 
    \[
    \sigma\tau(\alpha)=\sigma(\tau(\alpha))=\sigma(\omega_{\tau}\alpha)=\omega_\tau\sigma(\alpha)=\omega_{\tau}\omega_{\sigma}\alpha.
    \]
    Therefore $\Gal(K(\alpha)/K)$ is isomorphic to a subgroup
    of $G_n(K)$. In particular, $\Gal(K(\alpha)/K)$ is cyclic
    and $|\Gal(K(\alpha)/K)|$ divides $n$.
    
    
    Let $d=|\Gal(K(\alpha)/K)|$. Since $a=\alpha^n$, 
    \[
    \norm_{K(\alpha)/K}(\alpha)^n=\norm_{K(\alpha)/K}(a)=a^d.
    \]
    Thus $a^d\in K^n$, as $\norm_{K(\alpha)/K}(\alpha)\in K$. If $a^k\in K^n$, 
    say $a^k=c^n$ for some $c\in K$, then 
    \[
    c^n=a^k=(\alpha^n)^k=(\alpha^k)^n
    \implies 
    \alpha^k=c\omega\in K
    \]
    for some $\omega\in G_n(K)$. Thus $\alpha$ is a root of $X^k-\alpha^k\in K[X]$
    and hence $k\geq d$. 
    
    Note that $f(\alpha,K)=X^d-\alpha^d$. 
    
    Let $E/K$ be cyclic of degree $n$. Assume that  
    $\Gal(E/K)=\langle\sigma\rangle$.
    If $\omega$ is a primitive
    $n$-root of one, 
    \[
    \norm_{E/K}(\omega)=\omega^n=1.
    \]
    By Hilbert's theorem 90, 
    there exists $b\in E^{\times}$ such that 
    $\omega=\sigma(b)/b$. Thus
    $\sigma(b)=\omega b$ and hence 
    $\sigma^i(b)=\omega^i b$ for all $i\geq0$. Since 
    $|\{b,\sigma(b),\dots,\sigma^{n-1}(b)\}|=n$, 
    it follows that $E=K(b)$. Moreover, 
    \[
    \sigma(b^n)=\sigma(b)^n=(\omega b)^n=b^n
    \]
    and hence $b^n\in K$. This means that $E/K$ is a decomposition
    field of $X^n-b^n$. Note that $X^n-b^n$ is irreducible, as 
    $[E:K]=[K(b):K]=n$. 
\end{proof}

\begin{proposition}
    Let $K$ be a field of characteristic $p>0$. 
    \begin{enumerate}
        \item Let $a\in K$ and $f=X^p-X-a$. Then 
        $f$ is irreducible over $K$ or all the roots of 
        $f$ belong to $K$. 
        In the first case, if $b$ is a root of $f$, then 
        $K(b)/K$ is a cyclic extension of degree $p$. 
        \item Every cyclic extension of degree $p$ 
        is a splitting field of an irreducible
        polynomial of the form $X^p-X-a$. 
    \end{enumerate}
\end{proposition}

\begin{proof}
    We first prove 1). 
    Let $K_0$ be the prime field of $K$. Note that $K_0\simeq\Z/p$. 
    Let $b$ be a root of $f$ and let $x\in K_0$. 
    Then
    \[
    f(b+x)=(b+x)^p-(b+x)-a
    =(b^p-b-a)+(x^p-x)=0
    \]
    and thus $\{b+x:x\in K_0\}$ is the set of roots of $f$. Note that 
    $f'=-1$, so $f$ has no multiple roots. 
    
    We claim that if $b\not\in K$, then $f$ is irreducible. If 
    $f$ is not irreducible, then 
    $f=gh$ for some $g,h\in K[X]$ such that $0<\deg g<p$. There exists
    a subset $S$ of $K_0$ such that 
    $g=\prod_{x\in S}(X-(b+x))$ and hence
    \[
    |S|b+\sum_{x\in S}x=\sum_{x\in S}(b+x)\in K.
    \]
    This implies that $|S|b\in K$ and hence, since 
    $|S|\in K^{\times}$, it follows that 
    $b\in K$.
    
    Since $K(b)/K$ is a splitting field of a separable polynomial, 
    $K(b)/K$ is a Galois extension. Moreover, 
    $|\Gal(K(b)/K)|=[K(b):K]=p$ and hence 
    $\Gal(K(b)/K)$ is cyclic. 
    
    We now prove 2). Let $E/K$ be cyclic of degree $p$. Assume
    that $\Gal(E/K)=\langle\sigma\rangle$. Since
    $\trace_{E/K}(1)=p=0$, Hilbert's theorem 
    implies that there exists $b\in E$ such that 
    $\sigma(b)=b+1$. In particular, $b\not\in K$
    and thus $E=K(b)$. Moreover, since 
    \[
    \sigma(b^p-b)=\sigma(b)^p-\sigma(b)=(b+1)^p-(b+1)=b^p-b, 
    \]
    it follows that $b^p-b\in K$. Thus 
    $f(b,K)=X^p-X-(b^p-b)\in K[X]$. 
\end{proof}

\subsection{Symmetric polynomials}

Let $K$ be a field and $\{t_1,\dots,t_n\}$ be a commuting set of  independent variables.  
Let $E=K(t_1,\dots,t_n)$ and $f=\prod_{i=1}^n(X-t_i)\in E[X]$. 
Then
\[
f=X^n+\sum_{i=1}^n (-1)^i s_i X^{n-i},
\]
where 
\begin{align*}
    s_1 &= t_1+t_2+\cdots+t_n,\\
    s_2 &= \sum_{1\leq i<j\leq n}t_it_j,\\
    &\vdots\\
    s_n &= t_1t_2\cdots t_n.
\end{align*}
For example, 
\[
(X-t_1)(X-t_2)(X-t_3)=X^3-(t_1+t_2+t_3)X^2+(t_1t_2+t_2t_3+t_1t_3)X-t_1t_2t_3.
\]

The polynomials $s_1,s_2,\dots,s_n$ are known as the 
\emph{elementary symmetric polynomials} in the variables $t_1,\dots,t_n$. Note
that $\deg s_i=i$. 

Let $\sigma\in\Sym_n$ and 
\[
\alpha_{\sigma}\colon K[t_1,\dots,t_n]\to K[t_1,\dots,t_n],
\quad
t_i\mapsto t_{\sigma(i)}\quad\text{for all $i$}.
\]
Then $\alpha_{\sigma}$ is a bijective homomorphism of $K$-algebras. In fact, 
$\alpha_{\sigma}^{-1}=\alpha_{\sigma^{-1}}$. Note that
\[
\alpha_{\sigma}(h(t_1,\dots,t_n))=h(t_{\sigma(1)},\dots,t_{\sigma(n)}).
\]
Since $\alpha_{\sigma}$ is injective, 
it induces an element $\widehat{\sigma}\in\Gal(E/K)$ given by
\[
    \widehat{\sigma}\left(\frac{h}{g}\right)
    =\frac{\alpha_{\sigma}(h)}{\alpha_{\sigma}(h)}.
\]
The map $\Sym_n\to\Gal(E/K)$, $\sigma\mapsto\widehat{\sigma}$, 
is an injective group homomorphism. Thus
\[
\{\widehat{\sigma}:\sigma\in\Sym_n\}\simeq\Sym_n.
\]

\begin{definition}
    Let $g\in K[t_1,\dots,t_n]$. Then $g$ is \emph{symmetric} 
    if $\widehat{\sigma}(g)=g$ for all $\sigma\in\Sym_n$. 
\end{definition}
    
We write $P$ to denote the set of symmetric polynomials in $K[t_1,\dots,t_n]$. 
Clearly,
$P$ is a subalgebra of $K[t_1,\dots,t_n]$. The following statements hold:
\begin{enumerate}
    \item $K\subseteq P$.
    \item $\sum_{i=1}^n t_i^r\in P$ for all $r\geq1$.
    \item $s_i\in P$ for all $i$.
    \item $K(P)\subseteq\prescript{G}{}{E}$, where $G=\{\widehat{\sigma}:\sigma\in\Sym_n\}$.  
\end{enumerate}

Let $F=K(s_1,s_2,\dots,s_n)$. Then $E/F$ is a Galois extension, as 
it is a splitting field of $f$. 

\begin{proposition}
    $[E:F]\leq n!$.
\end{proposition}

\begin{proof}
    We proceed by induction on $n$. The case $n=1$ is clear, as $E=F$. Assume
    that $n>1$. Let $u_1,\dots,u_{n-1}$ be the elementary symmetric
    polynomials in $t_1,\dots,t_{n-1}$. Then
    \[
    s_i=u_i+t_nu_{i-1}
    \]
    for all $i\in\{1,\dots,n\}$, where $u_0=1$ and $u_n=0$. Note that
    $u_1=s_1-t_n$ and $u_i=s_i-t_nu_{i-1}$ for all $i$. Since
    $K(s_1,\dots,s_n,t_n)=K(u_1,\dots,u_{n-1},t_n)$, 
    \begin{align*}
    &F(t_n)=K(u_1,\dots,u_{n-1},t_n)=K(t_n)(u_1,\dots,u_{n-1})
    \shortintertext{and}
    &[E:F]=[E:F(t_n)][F(t_n):F]\leq n[E:F(t_n)].
    \end{align*}
    
    Note that $E=K(t_1,\dots,t_n)=K(t_n)(t_1,\dots,t_{n-1})$. By the
    inductive hypothesis, 
    \[
    [E:F(t_n)]\leq (n-1)!
    \]
    and hence 
    $[E:F]\leq n!$. 
\end{proof}

\begin{theorem}
    $\prescript{G}{}{E}=F$.
\end{theorem}

\begin{proof}
    By Artin's theorem,
    \[
    \left[\prescript{G}{}{E}:F\right]=\frac{[E:F]}{\left[E:\prescript{G}{}{E}\right]}
    \leq\frac{n!}{\left[E:\prescript{G}{}{E}\right]}=1
    \]
    and hence $\prescript{G}{}{E}=F$.
\end{proof}

\begin{exercise}
    Prove that $\Gal(E/F)\simeq\Sym_n$.
\end{exercise}

\begin{exercise}
    Prove that $\{s_1,\dots,s_n\}$ is algebraically independent over $K$. 
\end{exercise}

\begin{exercise}
    Prove that every symmetric polynomial in $t_1,\dots,t_n$ can be written as a rational
    fraction in $s_1,\dots,s_n$. 
\end{exercise}

