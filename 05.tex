\chapter{}


% pag 21 example

\begin{definition}
    Let $E/K$ be an algebraic extensions and $C$ be an algebraic closure of $K$. 
    Then $E/K$ is \textbf{normal} if 
    $\sigma(E)\subseteq E$ for all $\sigma\in\Hom(E/K,C/K)$. 
\end{definition}

Note that $\sigma(E)\subseteq E$ in the previous definition
is equivalent to $\sigma(E)=E$. 

\begin{example}
    The extension $\Q(\sqrt[3]{2})/\Q$ is not normal. Why?
\end{example}

Some trivial examples of normal extensions: $K/K$ is normal
and if $C$ is an algebraic closure of $K$, then $C/K$ is normal. 

\begin{example}
    The extension $\Q(\sqrt{2})/\Q$ is normal. In fact, 
    every extension generated by algebraic elements of degree two is normal. 
\end{example}

\begin{exercise}
    Let $\xi$ be a primitive cubic root of one. Then 
    $\Q(\sqrt[3]{2},\xi)/\Q$ is normal. 
\end{exercise}

The following result is useful but technical, that is why we leave the proof
as an exercise. 

\begin{exercise}
    Prove that the previous definition depends on $E$ and not on the
    algebraic closure $C$. 
\end{exercise}

Some properties:

\begin{proposition}
    Let $E/K$ be a normal extension and $f\in K[X]$ be an irreducible polynomial
    that admits a root $x$ in $E$. Then $f$ factorizes
    linearly in $E$.
\end{proposition}

\begin{proof}
    We may assume that $f$ is monic. Let $C/K$ be an algebraic closure of $K$ containing $E$. 
    Let $y$ be a root of $f$ in $C$. Since $f=f(x,K)=f(y,K)$, 
    it follows that $y=\sigma(x)$ for some $\sigma\in\Gal(C/K)$. Since 
    $E/K$ is normal, $\sigma|_E\colon E\to C$ is an automorphism of $E/K$, that is
    $\sigma(E)\subseteq E$. In particular, $y\in E$. 
\end{proof}

Let $K\subseteq F\subseteq E$ be a tower of fields. 
If $E/K$ is normal, then $E/F$ is normal. However, 
Note that $E/K$ normal does not imply $F/K$ normal, as this would imply 
that every extension is normal. Moreover, 
$E/F$ normal and $F/K$ normal do not imply $E/K$ normal.
    
\begin{example}
The extensions $\Q(\sqrt[4]{2})/\Q(\sqrt{2})$ and $\Q(\sqrt{2})/\Q$ are both
normal, but $\Q(\sqrt[4]{2})/\Q$ is not normal, 
as the roots of $X^4-2$ are
$\sqrt{2}$, $-\sqrt{2}$, $\sqrt{2}i$ and $-\sqrt{2}i$.
\end{example}


Recall that if $C$ is an algebraic closure of $K$ and $x\in C$,
then 
\[
f(x,K)=\prod(X-y)^m,
\]
where the product is taken over all 
$y\in O_{\Gal(C/K)}(x)$. 
If $E/K$ is normal and $x\in E$, then there exists $m$ such that 
\[
f(x,K)=\prod(X-y)^m,
\]
where the product is taken over all 
$y\in O_{\Gal(E/K)}(x)$. 

\begin{proposition}
    Let $E/K$ and $F/K$ be extensions. If $F/K$ 
    is normal, then $EF/E$ is normal.
\end{proposition}

\begin{proof}
    Let $C$ be an algebraic closure of $E$ containing $EF$. 
    Let $\sigma\in\Hom(EF/E,C/E)$. We claim that $\sigma(EF)=EF$. Let 
    \[
    \overline{K}=\{x\in C:x\text{ is algebraic over $K$}\}.
    \]
    Then $\overline{K}$ is an algebraic closure over $K$ and $F\subseteq\overline{K}$. 
    Since $F/K$ is normal and $\sigma|_F\in\Hom(F/K,\overline{K}/K)$, 
    it follows that $\sigma(F)=F$. If $z\in EF$, then
    $z=\sum_{i=1}^m e_if_i$ for some $e_1,\dots,e_m\in E$ and 
    $f_1,\dots,f_m\in F$. Since $\sigma(e_i)=e_i$ for all $i$,  
    \[
    \sigma(z)=\sum_{i=1}^m\sigma(e_i)\sigma(f_i)=\sum_{i=1}^m e_i\sigma(f_i)\in EF.\qedhere 
    \]
\end{proof}

\begin{proposition}
    Let $E/K$ be an algebraic extension. Then 
    $E/K$ is normal if and only if $E/K$ is the decomposition field
    of a family of polynomials of $K[X]$ of positive degree.
\end{proposition}

\begin{proof}
    Let $G=\Gal(E/K)$. If $x\in E$ and $f(x,K)=\prod_{y\in O_G(x)}(X-y)^m$, 
    then $f(x,K)$ factorizes linearly in $E[X]$. Thus 
    $E/K$ is a decomposition field of the family 
    $\{f(x,K):x\in E\}$. 
    Conversely, assume that $E/K$ is a decomposition field of the family 
    $\{f_i:i\in I\}$. Then $E=K(S)$ where $S$ is the set of roots
    of the polynomials $f_i$. Let $C/K$ be an algebraic closure
    of $K$ that contains $E$ and let $\sigma\in\Hom(E/K,C/K)$. Let $x\in S$. 
    Then $x$ is a root of some $f_j=\sum a_kX^k$. Since $f_j(x)=0$, 
    it follows that $\sigma(x)$ is a root of $f_j$, as 
    \[
    f_j(\sigma(x))=\sum a_k\sigma(x)^k
    =\sum\sigma(a_k)\sigma(x^k)
    =\sigma\left(\sum a_kx^k\right)=\sigma(0)=0.
    \]
    Hence $\sigma(E)\subseteq E$. 
\end{proof}

\section{Dedekind's theorem}

Note that every extension homomorphism $E/K\to F/K$ is, in particular, 
a $K$-linear map $E\to F$, that is
\[
\Hom(E/K,F/K)\subseteq\Hom_K(E,F).
\]
If $F/K$ is an extension and $V$ 
is a $K$-vector space, the set
$\Hom_K(E,F)$ of $K$-linear maps
is a vector space over $F$ with
$(a\cdot f)(v)=af(v)$ for $a\in F$, $f\in\Hom_K(E,F)$ and $v\in V$. 

\begin{exercise}
\label{xca:dim}
    Prove that $\dim_F\Hom_K(V,F)\geq\dim_KV$. Moreover, if 
     $\dim_KV<\infty$, then $\dim_F\Hom_K(V,F)=\dim_KV$.
\end{exercise}

If $V$ is a vector space and $S$ is a (possibly infinite) subset of $V$, 
then $S$ is linearly independent if every finite subset of $S$ is linearly independent. 

\begin{theorem}[Dedekind]
\index{Dedekind's theorem}
Let $E/K$ and $F/K$ be extensions and let 
$\{\varphi_i:i\in I\}$ 
be a subset of
$\Hom(E/K,F/K)$, i.e. a 
family of extension homomorphisms. Assume that 
$\varphi_i\ne \varphi_j$ if $i\ne j$. Then 
the subset $\{\varphi_i:i\in I\}\subseteq\Hom_K(E,F)$ 
is linearly independent over $F$. 
\end{theorem}

\begin{proof}
    Assume it is not. Let $\{\varphi_1,\dots,\varphi_n\}$ 
    be linearly dependent over $F$ with $n$ minimal. Clearly, $n>1$. 
    We may assume that 
    \begin{equation}
        \label{eq:Dedekind1}
        \sum_{i=1}^n a_i\varphi_i=0
    \end{equation}
    for some $a_1,\dots,a_n\in F$ all different from zero. 
    Let
    $z\in E\setminus\{0\}$ be such that $\varphi_1(z)\ne\varphi_2(z)$. If $x\in E$, then
    \[
    0=\left(\sum_{i=1}^na_i\varphi_i\right)(xz)=\sum_{i=1}^na_i\varphi_i(xz)
    =\sum_{i=1}^na_i\varphi_i(x)\varphi_i(z)
    =\left(\sum_{i=1}^n (a_i\varphi_i(z))\varphi_i\right)(x).
    \]
    Thus 
    \begin{equation}
        \label{eq:Dedekind2}
        \sum_{i=1}^n (a_i\varphi_i(z))\varphi_i=0.
    \end{equation}
    Since $\sum_{i=1}^na_i\varphi_i=0$ and $\varphi_1(z)\ne0$, 
    subtracting \eqref{eq:Dedekind1} and \eqref{eq:Dedekind2} we obtain that  
    \[
    a_1\varphi_1+a_2\frac{\varphi_2(z)}{\varphi_1(z)}\varphi_2+\cdots+a_n\frac{\varphi_n(z)}{\varphi_1(z)}\varphi_n=0.
    \]
    Thus
    \[
    \left(a_2-a_2\frac{\varphi_2(z)}{\varphi_1(z)}\right)\varphi_2
    +\cdots+\left(a_n-a_n\frac{\varphi_n(z)}{\varphi_1(z)}\right)\varphi_n=0.
    \]
    Since $a_n\ne 0$ and $\varphi_2(z)\ne\varphi_1(z)$, 
    the scalar $a_2-a_2\frac{\varphi_2(z)}{\varphi_1(z)}\ne 0$ and hence 
    $\{\varphi_2,\dots,\varphi_n\}$ is linearly dependent, a contradiction. 
\end{proof}

If $E/K$ and $F/K$ are extensions, 
let $\gamma(E/K,F/K)=|\Hom(E/K,F/K)|$. 

\begin{exercise}
Prove the following statements:
\begin{enumerate}
    \item $\gamma(E/K,F/K)\leq\dim_F\Hom_K(E,F)$.
    \item If $[E:K]<\infty$, then $\gamma(E/K,F/K)\leq[E:K]$. 
    \item If $x$ is algebraic over $K$, then $\gamma(K(x)/K,F/K)\leq\deg(x,K)$.
\end{enumerate}
\end{exercise}

If $C$ is an algebraic closure of $K$,
then we define $\gamma(E/K)=\gamma(E/K,C/K)$. This definition does
not depend on the algebraic closure. 

\begin{exercise}
\label{xca:gamma_C}
    If $C$ and $C_1$ are algebraic closures of $K$, then
    \[
    |\Hom(E/K,C/K)|=|\Hom(E/K,C_1/K)|.
    \]
\end{exercise}

\begin{proposition}
\label{pro:gamma_orbit}
    Let $C$ be an algebraic closure of $K$ and $G=\Gal(C/K)$. 
    If $x\in C$, then $\gamma(K(x)/K)=|O_G(x)|$. 
\end{proposition}

\begin{proof}
    If $\sigma\in\Hom(K(x)/K,C/K)$, then there exists $\phi\in G$ such that
    $\phi|_{K(x)}=\sigma$. Thus $\sigma(x)=\phi(x)\in O_G(x)$. Conversely,
    if $y\in O_G(x)$, then there exists $\tau\in G$ such that
    $y=\tau(x)$. Hence $\tau|_{K(x)}\in\Hom(K(x)/K,C/K)$ and 
    $\tau|_{K(x)}(x)=y$. In particular, $\gamma(K(x)/K)$ divides $\deg(x,K)$. 
\end{proof}


\begin{exercise}
If $E/K$ is finite, then $|\Gal(E/K)|\leq [E:K]$. Moreover, 
$E/K$ is normal if and only if $|\Gal(E/K)|=\gamma(E/K)$. 
\end{exercise}

\chapter{}

If $t\colon A\to B$ is a surjective map, then 
$a\sim a_1\Longleftrightarrow t(a)=t(a_1)$ 
defines an equivalence relation on $A$. The set $\overline{A}$ 
of equivalence classes is in bijective correspondence with $B$,
$\overline{A}\to B$, $\overline{a}\mapsto t(a)$. 
Moreover, if $|t^{-1}(\{b\})|=m$ for all $b\in B$, then 
$|A|=m|\overline{A}|=m|B|$. 

\begin{proposition}
    Let $E/K$ be algebraic and $F/K$ be a subextension such that 
    $E/F$ is finite. Then $\gamma(E/K)=\gamma(E/F)\gamma(F/K)$. 
\end{proposition}

 \begin{proof}
    Assume that $E=F(x)$. Let $f=f(x,F)=\sum b_iX^i$ and
    let $G=\Gal(E/F)$. Let $C$ be an algebraic closure of $K$ containing $E$. 
    The map
    \[
    \lambda\colon \Hom(E/K,C/K)\to\Hom(F/K,C/K),\quad
    \sigma\mapsto\sigma|_F,
    \]
    is well-defined. It is surjective: if $\varphi\in\Hom(F/K,C/K)$, then $\varphi\colon F\to C$ is, 
    in particular, a field homomorphism. Since $E/F$ is algebraic, by Proposition \ref{pro:Artin} 
    there exists a field homomorphism 
    $\sigma\colon E\to C$ such that $\sigma|_F=\varphi$. Since $\sigma|_K=\varphi|_K=\id$, in particular 
    $\sigma\in\Hom(E/K,C/K)$. 
    
    For $\varphi\in\Hom(F/K,C/K)$,  
    \[
    \lambda^{-1}(\{\varphi\})=\{\sigma\in\Hom(E/K,C/K):\sigma|_F=\varphi\}
    \]
    and let $R_\varphi$ be the set of roots (in $C$) of the polynomial $\overline{\varphi}(f)=\sum\varphi(b_i)X^i$. 
    
    \begin{claim}
    The map $\alpha\colon \lambda^{-1}(\{\varphi\})\to R_{\varphi}$, $\sigma\mapsto\sigma(x)$, is well-defined. 
    \end{claim}
    
    We need to show that $\sigma(x)$ is a root of $\overline{\varphi}(f)$:
    \begin{align*}
    \overline{\varphi}(f)(\sigma(x))&=\sum \varphi(b_i)\sigma(x)^i
    =\sum\sigma(b_i)\sigma(x^i)\\
    &=\sum\sigma(b_ix^i)=\sigma\left(\sum b_ix^i\right)=\sigma(f(x))=\sigma(0)=0.
    \end{align*}
    
    \begin{claim}
    The map $\beta\colon R_{\varphi}\to \lambda^{-1}(\{\varphi\})$, $y\mapsto\sigma_y$, 
    where $\sigma_y(z)=\overline{\varphi}(h)(y)$
    if $z=h(x)$, is well-defined. 
    \end{claim}
    
    We need to show that if $z=h(x)$ and 
    $z=h_1(x)$ for some $h,h_1\in F[X]$, then 
    $\overline{\varphi}(h)(y)=\overline{\varphi}(h_1)(y)$. 
    The assumptions imply that 
    $(h-h_1)(x)=0$ and hence $f$ divides $h-h_1$. Since
    $\overline{\varphi}$ is a ring homomorphism, 
    $\overline{\varphi}(f)$ divides $\overline{\varphi}(h)-\overline{\varphi}(h_1)$. 
    This implies $(\overline{\varphi}(h)-\overline{\varphi}(h_1))(y)=0$. We also need to show that 
    $\sigma_y|F=\varphi$: if $f\in F$, then 
    write $f=fX^0\in F[X]$. Thus 
    $\sigma_y(f)=\overline{\varphi}(fX^0)(y)=\varphi(f)\in C$. 
    We now left as an exercise to prove that $\sigma_y\in\Hom(E/K,C/K)$. 
    
    \begin{claim}
        $|\lambda^{-1}(\{\varphi\})|=|R_\varphi|$. 
    \end{claim}
    
    For this we need to show that $\beta$ is
    the inverse of $\alpha$, that is 
    $\alpha\circ\beta=\id$ and $\beta\circ\alpha=\id$. 
    To prove that $\beta\circ\alpha=\id$ 
    let $\sigma$ be such that $\sigma|_F=\varphi$. 
    Then $y=\sigma(x)\in R_\varphi$. Let
    $z=h(x)=\sum a_ix^i\in F[x]=E$. Then  
    \[
    \overline{\varphi}(h)(y)=\sum\varphi(a_i)y^i=\sum\sigma(a_i)y^i
    =\sigma\left(\sum a_ix^i\right)=\sigma(y).
    \]
    Conversely, if $y\in R_\varphi$, then
    \[
    \alpha(\sigma_y)=\sigma_y(x)=y,
    \]
    as $\sigma_y(x)=\overline{\varphi}(X)(y)=y$.
    
    \begin{claim}
        If $\phi\in\Gal(C/K)$ is such that $\phi|_F=\varphi$, then 
        $O_{\Gal(C/K)}(x)=\phi^{-1}(R_\varphi)$.
    \end{claim}

    Let us first prove $O_{\Gal(C/K)}(x)\supseteq \phi^{-1}(R_\varphi)$.
    If $y\in R_{\varphi}$, 
    then 
    \begin{align*}
    f(\phi^{-1}(y))&=\sum b_i\phi^{-1}(y^i)=\phi^{-1}\left(\sum\phi(b_i)y^i\right)\\
    &=\phi^{-1}\left(\sum\varphi(b_i)y^i\right)=\phi^{-1}\overline{\varphi}(f)(y))=\phi^{-1}(0)=0.
    \end{align*}
    Now we prove $O_{\Gal(C/K)}(x)\subseteq\phi^{-1}(R_\varphi)$.
    Let $z\in O_{\Gal(C/K)}(x)$ and $y\in C$ be such that 
    $\phi^{-1}(y)=z$. Then $\overline{\varphi}(f)(y)=0$, as
    \begin{align*}
    \overline{\varphi}(f)(y)&=\sum\varphi(b_i)y^i\\
    &=\sum\varphi(b_i)\phi(z^i)
    =\sum\phi(b_i)\phi(z^i)=\phi\left(\sum b_iz^i\right)=\phi(f(z))=\phi(0)=0.
    \end{align*}
    
    \medskip
    It follows that $|\lambda^{-1}(\varphi)|=|O_{\Gal(C/K)}(x)|$ for
    all $\varphi$. By using the argument
    before the proposition, 
    \begin{align*}
    \gamma(E/K)&=|\Hom(E/K,C/K)|\\
        &=|O_{\Gal(C/K)}(x)||\Hom(F/K,C/K)|\\
        &=|O_{\Gal(C/K)}(x)|\gamma(F/K).
    \end{align*}
    Since $\gamma(K(x)/K)=|O_{\Gal(C/K)}(x)|$ by Proposition \ref{pro:gamma_orbit}, the claim follows. 
    
    \medskip 
    For the general case we assume that $E=F(x_1,\dots,x_n)$. We proceed
    by induction on $n$. If $n=0$, then $E=F$ and the result is trivial. 
    If $n>0$, let $L=F[x_1,\dots,x_{n-1}]$ and $E=L(x_n)$. The previous
    case applied in a different setting 
    implies that $\gamma(E/F)=\gamma(E/L)\gamma(L/F)$. By the inductive 
    hypothesis, $\gamma(L/K)=\gamma(L/F)\gamma(F/K)$. Thus 
    \[
    \gamma(E/F)\gamma(F/K)=\gamma(E/L)\gamma(L/F)\gamma(F/K)
    =\gamma(E/L)\gamma(L/K)=\gamma(E/K),
    \]
    again using the previous case. 
\end{proof}

\topic{Separable extensions}

\begin{definition}
    Let $E/K$ be an algebraic extension and $x\in E$. Then
    $x$ is \textbf{separable} over $K$ if $x$ is a simple root
    of $f(x,K)$. 
\end{definition}

An algebraic extension $E/K$ is \textbf{separable} 
if every $x\in E$ is separable over $K$. Clearly, $K/K$ is separable. 

\begin{exercise}
    Prove that 
    an element $x$ is separable over $K$ if and only if $x$ is a simple root
    of a polynomial with coefficients in $K$. 
\end{exercise}

If $F/K$ is a subextension of $E/K$ and $x\in E$ is separable over $K$, then
$x$ is separable over $F$. 

\begin{exercise}
     If $C$ is an algebraic closure of $K$, $x\in C$ and $G=\Gal(C/K)$
    Prove that the following statements are equivalent:
    \begin{enumerate}
        \item $x$ is separable over $K$.
        \item Every $y\in O_G(x)$ is separable over $K$.
        \item $\gamma(K(x)/K)=[K(x):K]=\deg f(x,K)$. 
    \end{enumerate}
\end{exercise}

Let $K$ be any field and $g\in K[X]$. Let $z$ be a root of $g$. 
Then $z$ is a multiple root of $g$ if and only if $z$ is a root of $g'$. 

\begin{exercise}
Prove that if $K$ has characteristic zero or $K$ is finite, then 
every algebraic extension of $K$ is separable. 
\end{exercise}

A consequence: 
Let $E/K$ be a finite extension. Then $E/K$ is separable
if and only if $\gamma(E/K)=[E:K]$. 

\begin{example}
    Let $E=\Q(\sqrt{2},\sqrt{3})$. Then 
    $[E:\Q]=4$ and 
    $\Gal(E/Q)\simeq C_2\times C_2$. The extension $E/Q$ is normal, 
    as it is the decomposition field of $(X^2-2)(X^2-3)$ and 
    it is separable as $\Q$ has characteristic zero. 
    % If $\sigma\in\Gal(E/\Q)$, then
    % $\sigma(\sqrt{2})\in\{-\sqrt{2},\sqrt{2}\}$ and 
    % $\sigma(\sqrt{3})\in\{-\sqrt{3},\sqrt{3}\}$. 
\end{example}

\begin{example}
    Let $E$ be a decomposition field of $X^4-2$ over $\Q$. 
    Then $E/\Q$ is normal and separable. Note that
    $E=\Q(\sqrt[4]{2},i)$, so $[E:\Q]=8=|\Gal(E/\Q)|$. 
    
    Let us compute
    $\Gal(E/\Q)$. If $\sigma\in\Gal(E/\Q)$, then 
    $\sigma(\sqrt[4]{2})\in\{\sqrt[4]{2},-\sqrt[4]{2},\sqrt[4]{2}i,-\sqrt[4]{2}i\}$ and 
    $\sigma(i)\in\{-i,i\}$. Two examples are 
    \[
    \alpha\colon\begin{cases}
    \sqrt[4]{2}\mapsto\sqrt[4]{2}i,\\
    i\mapsto i,
    \end{cases}
    \quad
    \beta\colon\begin{cases}
    \sqrt[4]{2}\mapsto\sqrt[4]{2},\\
    i\mapsto -i.
    \end{cases}
    \]
    It follows that 
    $\Gal(E/\Q)$ is isomorphic to the group $\langle\alpha,\beta\rangle$, which turns out to be
    isomorphic to the dihedral group
    of eight elements. 
\end{example}

Another consequence: If $E=K(S)$, then $E/K$ is separable if and only if
every $x\in S$ is separable over $K$. One first does the case $E=K(x)$ 
and then proceed by induction. 

\begin{exercise}
\label{xca:separable1}
    Let $K\subseteq F\subseteq E$ be a tower of fields. Prove that 
    if $E/K$ is separable, then $F/K$ and $E/F$ are separable. 
\end{exercise}

\begin{exercise}
\label{xca:separable2}
    Let $E/K$ and $F/K$ be extensions. Prove that if $E/K$ is separable, 
    then $EF/E$ is separable. 
\end{exercise}

% This follows because $EF/E$ is algebraic and 
% $EF/E=E(F)/E$ is generated by separable elements.

