\chapter{}

\topic{Galois extensions}

Let $E/K$ be an algebraic extension. Assume that $E=K(S)$ and
let $C$ be an algebraic closure of $K$ containing $E$. Let 
\[
T=\{y\in C:y\text{ is a root of $f(x,K)$ for some $x\in S$}\}
\]
and let $L=K(T)$. Then $E\subseteq L$, as $S\subseteq T$. The extension
$L/K$ is normal, as $L/K$ is a decomposition field of the family $\{f(x,K):x\in S\}$. 
Moreover, $L$ is the smallest normal extension of $K$ containing $E$. The field
$L$ is the \textbf{normal closure} of $E$ (with respect to $C$). 

\begin{exercise}
If $E/K$ is finite, then $L/K$ is finite
\end{exercise}

\begin{exercise}
If $E/K$ is separable, then $L/K$ is separable.
\end{exercise}

Let $E/K$ be an extension and $S\subseteq\Gal(E/K)$ be a subset. 
the set 
\[
    \prescript{S}{}{E}=\{x\in E:\sigma(x)=x\text{ for all $\sigma\in S$}\}
\]
is a subfield of $E$ that contains $K$. The subfield $\prescript{S}{}{E}$
is known as the \textbf{fixed field} of $S$. 

\begin{definition}
    \index{Extension!Galois}
    Let $E/K$ be an algebraic extension and $G=\Gal(E/K$). 
    Then $E/K$ is a \textbf{Galois extension} if $\prescript{G}{}{E}=K$. 
\end{definition}

Clearly, $K/K$ is a Galois extension. 
Note that $\Q(\sqrt[3]{2})/\Q$ is not a Galois extension. Why?

\begin{exercise}
    Prove that $\Q(\sqrt{2},\sqrt{3})/\Q$ is a Galois extension. 
\end{exercise}

\begin{exercise}
If the characteristic of $K$ is different from two, 
then every quadratic extension of $K$ is a Galois extension. 
\end{exercise}

\begin{exercise}
    Let $E/K$ be an algebraic extension and $G=\Gal(E/K)$. Let
    $F=\prescript{G}{}{E}$. Prove that $\Gal(E/F)=G$ and hence $E/F$ is a Galois extension. 
\end{exercise}

\begin{proposition}
    Let $E/K$ be an algebraic extension. Then $E/K$ is a Galois extension
    if and only if $E/K$ is normal and separable. 
\end{proposition}

\begin{proof}
    Let $G=\Gal(E/K)$. Let us first assume that $E/K$ is Galois. Let $x\in E$
    and $f_x=\prod_{y\in O_G(x)}(X-y)\in E[X]$. If $\varphi\in G$, then 
    $\overline{\varphi}(f_x)=\prod_{y\in O_G(x)}(X-\varphi(y))$... (pag70)
\end{proof}

\begin{definition}
Let $K$ be a field and $f\in K[X]$. Then $f$ is \textbf{separable}
if all roots of $f$ are simple (in some algebraic closure of $K$). 
\end{definition}

\begin{proposition}
    Let $E/K$ be a finite extension. Then $E/K$ is a Galois extension 
    if and only if $E$ is a decomposition field over $K$ 
    of a separable polynomial $f\in K[X]$. 
\end{proposition}

\begin{proof}
    
\end{proof}

In the previous case, $\Gal(E/K)$ is known as the \textbf{Galois group}
of the polynomial $f$. The notation is $\Gal(f,K)$. 

\begin{theorem}[Artin]
\index{Artin's theorem}
\label{thm:ArtinGalois}
    Let $E$ be a field and $G$ be a finite group of automorphisms of $E$. 
    If $K=\prescript{G}{}{E}$, then $E/K$ is a Galois extension,
    $[E:K]=|G|$ and $\Gal(E/K)=G$. 
\end{theorem}

Before proving the theorem, we need a lemma.

\begin{lemma}
    Let $E/K$ be a separable extension such that $\deg(x,K)\leq m$
    for all $x\in E$. Then $E/K$ is finite and $[E:K]\leq m$. 
\end{lemma}

\begin{proof}
   Let $z\in E$ be of maximal degree. If $x\in E$, 
   then $K(x,z)/K$ is separable. Then $K(x,z)=K(y)$ for some $y$. 
   It follows that 
   \[
   K(z)\subseteq K(x,z)=K(y).
   \]
   Since 
   $\deg(z,K)\leq\deg(y,K)$, it follows that
   $\deg(z,K)=\deg(y,K)$ and hence 
   $K(y)=K(z)$. In particular, $x\in K(z)$ and
   therefore $E=K(z)$. 
\end{proof}

Now we are ready to prove Artin's theorem: 

\begin{proof}[Proof of Theorem \ref{thm:ArtinGalois}]
    Note that $G\subseteq\Gal(E/K)$. Let $x\in E$ and 
    \[
    f_x=\prod_{y\in O_G(x)}(X-y).
    \]
    Since $f_x\in K[X]$, it follows
    that $E/K$ is normal and separable, so $E/K$ is a Galois extension. Moreover, 
    \[
    \deg(x,K)\leq \deg f_x=|O_G(x)|\leq |G|.
    \]
    By the previous lemma, $E/K$ is finite and $[E:K]\leq |G|$. This
    implies that
    $|G(E/K)|=[E:K]\leq |G|$ and hence $|G(E/K)|=|G|$. 
\end{proof}

\topic{Galois' correspondence}

\begin{theorem}[Galois]
\index{Galois' theorem}
    Let $E/K$ be a finite Galois extension and $G=\Gal(E/K)$. 
    There exists a bijective correspondence
    \[
    \{F:K\subseteq F\subseteq E\text{ subfields}\}\to 
    \{\text{subgroups of $G$}\}
    \]
    The correspondence is given by $F\mapsto G(E/F)$ and 
    $\prescript{S}{}{E}\mapsfrom S$. Moreover, 
    normal subextensions of $E/K$ correspond 
    to normal subgroups of $G$. 
\end{theorem}

