\chapter{}

\topic{The fundamental theorem of algebra}

We now present an easy proof of the fundamental theorem 
of algebra based on the ideas of Galois Theory. 
We need the following well-known facts:
\begin{enumerate}
\item Every real polynomial of odd degree admits a real root. This means that $\R$ 
does not admit extension of odd degree $>1$. 
\item Every complex number admits a square root in $\C$. This means that $\C$ 
does not admit degree-two extensions.
\end{enumerate}

\begin{theorem}
The field $\C$ is algebraically closed.
\end{theorem}

\begin{proof}
    Let $E/\C$ be an algebraic finite extension. Then $E/\R$ 
    is finite separable extension of even degree. There exists a Galois
    extension 
    $L/\R$ such that $E\subseteq L$, so $[L:\R]$ is even. Let $G=\Gal(L/\R)$. 
    Then $|G|=2^ms$ for some odd number $s$. If $T$ is a 2-Sylow subgroup
    of $G$, 
    then there exists a subextension $F/\R$ of degree $s$. Since 
    $\R$ does not admit extensions of odd degree $>1$, $s=1$ and
    hence $G$ is a $2$-group. In particular, $|\Gal(L/\C)|=2^{m-1}$. If $m>1$, 
    let $U$ be a subgroup of $\Gal(L/\C)$ of order $2^{m-2}$. Then $U$ corresponds 
    to a subextension $L_1/\C$ of degree two, a contradiction. Hence $m=1$ 
    and $[L:\C]=1$, so $L=\C$ and $E=\C$. 
\end{proof}

\topic{Purely inseparable extensions}

Let $E/K$ be an algebraic extension. 
In page \ref{separable} we defined the 
\textbf{separable closure} of $K$ with respect to $E$ as 
the field 
\[
    F=\{x\in E:x\text{ is separable over }K\}.
\]
Note that $K\subseteq F\subseteq E$ 
and $F=K(F)$. Moreover, 
$F/K$ is separable and 
$E/F$ is a \textbf{purely inseparable} extension, meaning that
for every $x\in E\setminus F$, the polynomial $f(x,F)$ is not separable. 

The number $[E:F]$ is known as the \textbf{degree of inseparability} of $E/K$. 

Clearly, $E/K$ is separable if and only if $[E:F]=1$ and 
$E/K$ is purely inseparable if and only if $[E:F]=[E:K]$. 

\begin{proposition}
Let $K$ be a field of characteristic $p>0$ and
$E/K$ be an algebraic extension. The following statements are equivalent:
\begin{enumerate}
    \item $E/K$ is purely inseparable.
    \item If $x\in E$, then $x^{p^m}\in K$ for some $m\geq0$.
    \item If $x\in E$, then $f(x,K)=X^{p^m}-a$ for some $a\in K$ and $m\geq0$. 
    \item $\gamma(E/K)=1$. 
\end{enumerate}
\end{proposition}

\begin{proof}
    
\end{proof}


